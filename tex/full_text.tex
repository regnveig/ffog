\author{Эмиль Хельгасон}
\title{Цветы для бабушки}
\date{19.08.2017}
\maketitle

\tableofcontents

\chapter{Цветы для бабушки}

\section{Кошмар}

Вспомните свой самый худший кошмар.

Вспомнили?
Во всех подробностях?

Так вот, это ерунда.

Самый худший кошмар на свете выглядит достаточно безобидно.
Кошмар ограниченного мира.
Вариаций может быть огромное множество, и Варенику сегодня приснилась одна из них.
Звёздное небо, дом с пустыми окнами, занесённый тонким слоем снега переулок и пять могил на обочине.

Важная часть кошмара ограниченного мира "--- осознанность и кратковременная иллюзия всемогущества.
Да, я во сне, а значит, всё, что я могу придумать, может стать реальностью!
И только взлетев, распахнув крылья, ощутив биение воздуха в грудь, Вареник понял "--- не может.

Далёкие звёзды оказались простыми источниками света, подвешенными в пустоте.
Дом превратился в картонку с прорезями, а переулок "--- в единственное во всей этой карманной Вселенной место, где можно стоять и существовать.

Больше там ничего не было.

Вареник спустился к могилам и начал читать надгробия:

\begin{quote}
Владимир Владимирович Шенкерман (1986--2048)

Вероника Андреевна Сафронова-Зима (1989--2072)

Сергей Валерьевич Сафронов (2010--2045)

Лидия Валерьевна Сафронова (2008--2090)
\end{quote}

Последняя могила, принадлежащая <<Валерию Николаевичу Сафронову (1985--2028)>>, была пустой.
Кто-то положил на дно клетчатое одеяло, приглашающе отогнув уголок.

Что, если бы вы не смогли проснуться?

Варенику повезло "--- он проснулся.
Могилы и снег сменились теплом постели и живой, невероятно живой жены с огненно-рыжими волосами.
И даже мысль о сходстве карманной и реальной Вселенной его не посетила.
Ему вообще везло "--- и с мыслями, и с женой, и с самой жизнью "--- как и восьми миллиардам смертных, которым есть куда проснуться и есть куда умереть.

Впрочем, именно сегодня он не чувствовал себя везучим.
На утро было назначено очередное собеседование.

\section{Декаданс}

"--*Извините, вы нам не подходите.

Стандартная фраза.
Вареник слышал её уже в одиннадцатый раз.
Говорили на собеседованиях разное, но слышал он одно и то же.
Иногда проводивший собеседование эйчар даже начинал извиваться и кривиться, словно змея на сковороде;
Вареник знал "--- наступил момент, когда уже всё ясно, деньги отработаны, а желудок требует заслуженной чашки корпоративного кофе.

"--*Прощайте, "--- коротко ответил Вареник и, собрав свои бумаги, вышел из кабинета.

В этом офисе, как и в прочих других, царил декаданс.
Современные материалы, на поверку оказывающиеся отходами производства.
Запах парфюмерии, сквозь который пробивались ароматы непролеченных язв и убитых алкоголем почек.
Десятки людей с агрессивно-равнодушными взглядами, прячущими страх безработицы, ощущение собственной незначительности и два-три непогашенных кредита.

<<Может, оно и к лучшему>>, "--- решил про себя Вареник, ощущая тяжёлый взгляд охранника.
Тот крутил на пальце ключи.
Две тысячи лет назад жил человек, который с точно таким же выражением лица поигрывал кнутом.
Этого не было ни в одной хронике, ни в одном учебнике, но Вареник знал это так, словно видел своими глазами.

<<Прости, Вероника.
Хуёвый у тебя муж>>.

\section{Плохое предчувствие}

Предчувствие очередной неудачи преследовало Вареника с утра "--- с того самого момента, как он покинул мир с пятью могилами.

Вначале он вспомнил, что машина в автосервисе, и ехать придётся общественным транспортом.
Затем выяснилось, что интернет закончился в полночь, и узнать дорогу к офису компании можно только у точки бесплатного вайфая в двух остановках от дома.
Наконец, уже настроившись перехватить в кафе пирожок, он выяснил, что забыл в другой куртке кошелёк.
Налички было впритык на дорогу.

Офис находился на возвышенности;
ледяной ветер с Оби забирался под куртку и сбивал с ног.

Охрана сразу встретила его неприветливо.

"--*Я на собеседование.

"--*К кому?

<<В смысле "--- к кому?>>
В здании находилась только одна компания.

"--*Щас.

Вареник выудил телефон и начал искать СМС с данными компании.

"--*Фома Непомнящий, блядь, "--- плюнул охранник.
"--- Идёт, а к кому "--- не помнит.

"--*Максим Орлов, "--- сказал Вареник, проигнорировав реплику.

"--*Ну так и звоните ему.

"--*До собеседования ещё час, "--- сказал Вареник.
"--- На улице ветрено, и я думал, просто посижу здесь.

"--*Здесь вы не посидите, "--- обрадовал его охранник.
"--- Приходите ко времени.

Вареник кивнул и направился к выходу.

Райончик оказался так себе.
Автострада, автомойки, автосервисы, несколько клиник, пара задрипанных ювелирных салонов и билборд <<Откажись от наркотиков!>>.
Как известно, билборды и надписи на стенах всегда соответствуют местному контингенту.
Ни одного кафе в радиусе километра.

Местный продуктовый был на три четверти заставлен спиртным, одну восьмую прилавков занимали семечки "--- ещё один намёк на местный контингент.
Собственно съестное робко ютилось в уголке.
Вареник купил крохотную пачку вафель, чтобы хоть как-то заполнить пустоту в желудке.

Оставшиеся десять минут он провёл в странно современном цветочном магазине, чихая и делая вид, что рассматривает цветы.
Симпатичная продавщица на каждый чих говорила <<Будьте здоровы>> и понимающе вздыхала.

Всё это было зря.

\section{Забор и шлагбаум}

Выйдя из офисного здания, Вареник направился к метро.

К станции удалось выйти не сразу.
Дорожная развязка сыграла с пешеходом несколько ей одной понятных шуток, прежде чем милостиво вывела его на нужную улицу.
Вареник обрадовался и рванул через дворы, но радость оказалась преждевременной.
На пути встали несколько новостроек, окружённых глухими заборами;
шлагбаумы стояли, точно крохотные ландскнехты с огромными двуручными мечами.

<<Да какого хрена! "--- возмутился Вареник.
"--- Как вообще здесь выживать без автомобиля?!
Лабиринт ебаный\ldotst >>

Совсем некстати опять всплыл в памяти утренний сон.

Вареник ненавидел колючую проволоку и заборы с острыми пиками.
Эти вещи "--- не просто архитектурные ухищрения, не просто предупреждение;
это настоящее варварство, полное презрение к человеческим страданиям.
Ни одна страна не будет свободной, пока на её территории есть колючая проволока.
В России же даже в городах, предназначенных для проживания свободных людей, каждый второй дом, каждый промышленный или торговый объект ощерились острым металлом.
Пожалуй, единственное, что не имеет границ в России "--- стремление возводить границы.

"--*Э, сука, куда полез! "--- заорал кто-то вдалеке.
Вареник аккуратно перемахнул через шипастое навершие забора, пропустив реплику мимо ушей.

Законы, финансы, политика, военные зоны, частная собственность "--- удавки, наброенные на шею и не дающие сделать вдох ни на миллилитр больше, чем нужно для твоего выживания.
Живи, работай и даже не помышляй выйти за рамки.
И не думай, что эти рамки иллюзорны "--- нужно большое мужество, чтобы уйти за границы собственных финансов или привычек.

Ты не можешь построить дом там, где хочешь.
Ты не можешь выращивать культуры и питаться от земли.
Ты не можешь перемещаться по стране, не имея целой кипы бумаг "--- документов или валюты.

Вареник желал другого мира.
Но мир только один, что бы ни говорили на этот счёт теории мультивселенных.

\ldotst Из-за домов выглянул кусочек оживлённой дороги, а затем показалась и большая красная буква <<М>>.
Философия вылетела из головы Вареника так же быстро, как и появилась.

\section{Отголосок бури}

"--*Не взяли, "--- утвердительно сказала Вероника, едва лицо мужа показалось из-за двери.
"--- Иди кушать, через десять минут готово будет.

Вареник повиновался.

Едва он сел за стол, как зазвонил телефон.

"--*Варёныч, хай.
Как здравие?

"--*Жив, Киря, твоими молитвами, "--- лаконично ответил Вареник.
"--- Сам как?

"--*Да как-то вот, Варёныч.
Короче, ушёл я из <<Ангстрема>>, и Воля ушёл.
По твоим стопам.

Голос Кири выдавал его напряжение.

"--*Если ты хочешь занять, не могу помочь, "--- предупредил Вареник очевидный вопрос.
"--- Я как уволился, так и без работы сижу.

"--*Бля, братан, "--- опечалился Киря.
"--- Мож варик какой есть, чтоб по-быстрому?
Лавэ горит, Анька свалить грозится.
Я по кентам посмотрел "--- нихуя, у всех напряг.

"--*Напиши мне ВКонтакте, я тебе скину кое-что, "--- задумался Вареник.
"--- Вахтовая работа, я такую точно не потяну по здоровью.
Платят немного, но зато вагончик и жратва включены, с голоду не помрёшь.

"--*Заебись!
Я тебе прям щас маякну!
От души, Варёныч!

"--*Говно вопрос, обращайся.

"--*Это с твоего склада? "--- поинтересовалась Вероника.

"--*Ага, "--- кивнул Вареник.
"--- Там походу всё.

\section{<<Ангстрем>>}

Торговая компания <<Ангстрем>> владела складами в нескольких регионах.
Ходили слухи, что её хозяин давно уже перешёл на более выгодный бизнес, перекупив часть акций известных торговых сетей;
склады остались придатком "--- малоприбыльным и морально устаревшим.

"--*КОООСТЯ, БЛЯЯЯ!
Опять ты мне, сука, собрал бочку с порохом! "--- с этого крика кладовщика Володи начинался рабочий день Вареника.

Уже слегка поддатый Костя бормотал извинения и шёл исправлять косяки в сборке.
Ещё совсем нестарый мужик бухал как чёрт две трети жизни.
Костю увольняли три раза, и три раза он возвращался "--- идти ему было некуда, и заменить его было некем.

Киря же, в отличие от Кости, не косячил никогда.
Он кабанчиком метался по складу, молниеносно раскидывая брикеты и связки по коробам.

"--*Ну а хуле, "--- говорил Киря прочим, "--- хочешь жить "--- умей вертеться.

Киря замечательно умел вертеться.
Он получал самую большую зарплату для его должности, его регулярно объявляли работником месяца и выдавали поощрения.
Он имел безусловный авторитет среди мужиков благодаря надёжности и честности.
Работа на складе была его коньком, а должность комплектовщика <<Ангстрема>> "--- венцом карьеры.

Вареник не хотел вертеться "--- он хотел жить.
Посему же ничем не выделялся среди других.
Ему нравился запах коробок и хруст, с которым они разрывались.
Платили ему гораздо меньше Кири, но тоже неплохо, вовремя "--- безусловное преимущество крупной компании.
Впрочем, пятое чувство подсказывало "--- пора валить.

И оно не обмануло.
Спустя пять дней после увольнения Вареник узнал, что склад влетел на крупную сумму "--- больше двадцати пяти тысяч на каждого работника.
Непробитая сборка пропала вместе с фурой, водителем и кое-какими важными документами.
Долг менеджеры милостиво растянули на полгода, не оставляя и тени сомнения в великолепии этого бизнес-плана.

"--*Ты всё ещё грустишь? "--- спросила Вероника, погладив мужа по голове.

"--*Это удивительно? "--- улыбнулся Вареник, мешая суп ложкой.
Суп уже давно остыл.

"--*Нет, "--- ответила жена.
"--- И всё же ты выглядишь счастливее, чем когда ты работал на складе.
Ложись-ка спать.
Я сегодня ненадолго, вечером кино посмотрим с пиццей.

Вечером позвонил с благодарностями Киря, обещал занести пиво.
Он за полдня устроился на ту самую вахту на вполне достойных условиях.
Вареник слушал его гоповатую речь в трубке с некоторой завистью.

<<И почему я так не могу, а?>>

\section{Аркада}

Следующее собеседование было назначено на завтра.

На этот раз пришло очень много народа "--- молодые и не очень, тёртого вида мужички и такие же ушлые бабы.
Вареник, привыкший к собеседованиям тет-а-тет, смотрел на них с удивлением.
Затем пришла милая женщина-эйчар и долго говорила о бонусах, комбо и планах компании.
Правда, её глаза светились чем-то непонятным "--- не усталостью, не скукой, а скорее полной тревоги грустью.
Видимо, у неё тоже был план, который следовало выполнить.

После наступило время самопрезентации.
Тёртые мужички и ушлые бабы разом превратились в нежных овечек, уверяя девушку, что они будут самыми спокойными и продуктивными работниками.
Вареник тем временем прокручивал в голове услышанные цифры.
Вскоре, отчаявшись найти у себя способности к математике, он вышел в коридор и позвонил лучшему другу, Владимиру Шенкерману.

"--*Нахуй, "--- сразу сказал Шенкерман.
"--- Платить тебе будут копейки.
Двойная премия выплачивается за перевыполнение плана, то есть сразу урезай все бонусы вдвое.
Кроме того, целых десять пунктов, связанных между собой какой-то хитровыебанной связью, про которую она вам ничего не сказала!
Не, Вареник, лучше вали оттуда.
Это игра с неизвестными правилами, ты однозначно проиграешь.

"--*Да крупная же вроде компания! "--- возражал Вареник.

"--*Серьёзные дяди устанавливают нормальную базу, а не устраивают аркаду с бонусами.
Она сказала тебе размер базового оклада?
Сказала?

"--*МРОТ.

"--*Умножь его на два, если ты хороший работник.
Умножь его на полтора, если ты хороший работник, а на дворе две тысячи двенадцатый год и кризис.
Вот твоя зарплата.

"--*Ты думаешь, что кризис есть на самом деле?

"--*Какая разница, есть он или нет?
Важно, что о нём говорят.
Уровень твоей зарплаты зависит от трепла в телевизоре, а не от невидимой руки рынка.

Вареник извинился перед милой девушкой-эйчаром и ушёл, не дождавшись своей очереди.

\section{Шенкерман}

Владимир Шенкерман был другом Вареника с ранней юности.
Кажется, они учились на одной параллели, но прочие подробности их знакомства терялись в сумраке веков.
Да и какая разница "--- была это какая-то школьная пьянка, жили ли они в одном дворе или просто столкнулись в коридоре?
Разницы нет никакой.

Шенкерман был, по представлениям Вареника, <<серьёзным человеком>> "--- он работал девелопером в небольшой, но неплохо зарекомендовавшей себя фирме.
А ещё Шенкерман, по выражению Вареника, умел <<делать деньги из воздуха>>, то есть занимался фрилансом.
Пожалуй, единственное, что плохо получалось у работящего парня "--- личная жизнь.
Он знал, каким концом нужно тыкать в девушку букетом, он знал несколько достаточно оригинальных комплиментов, но это почему-то совсем не помогало.
Теми же, кто очевидно зарился на его достаток, он брезговал.

Впрочем, отношения у Шенкермана были "--- и даже длительные.
Однажды он разрывался аж между двумя девушками;
Варенику даже пришлось вмешаться, пока в дело не встрял алкоголь.
Всегда готовый разрулить любую ситуацию, Шенкерман был абсолютно беспомощен перед девушками и вином.

"--*Вот чем тебе нравится Светочка? "--- допытывался Вареник во время очередного сеанса психотерапии.

"--*Светочка хорошо готовит, с ней есть о чём поговорить, "--- отвечал Шенкерман после некоторого раздумья.

"--*А Олечка?

"--*Олечка картавит, "--- не задумываясь выпаливал влюблённый.

Вареник тяжко задумывался и замолкал.
Видимо, в представлении Шенкермана картавость если и не перекрывала все достоинства Светочки, то по крайней мере могла составить им серьёзную конкуренцию.

Светочка всё же выиграла соревнование с небольшим перевесом.
На целых два года Шенкерман выпал из эволюционного процесса.
Обстоятельств, по которым они разошлись, Вареник так и не узнал;
Шенкерман начинал плеваться при любом упоминании имени <<Света>>.

\section{Старые связи}

Работа нашла Вареника неожиданно.
Он случайно зашёл в кафе-столовую неподалёку от дома;
сети обслуживания <<С пылу с жару>> требовались сотрудники.

"--*Да мы вас знаем, вы же к нам с женой регулярно ходили в прошлом году! "--- радостно сказала менеджер.
"--- И девочки вас тоже помнят.
Приходите завтра, пара дней стажировки "--- и мы вас устраиваем.

\section{Ревность}

Так для Вареника началась новая жизнь, полная жратвы и адреналина.
На его взгляд, сочетание было идеальным.

Жратву Вареник ценил, особенно вкусную.
Наверное, это приходит к любому, кто хотя бы два месяца жизни сидел на крупах.
Или две недели лежал в реанимации с назогастральным зондом.
Приверженцы диет редко знают лицо настоящего голода.

"--*Так, что за дела? "--- возмутилась Вероника, когда муж впервые отказался вечером от супа.
"--- Ты любовницу завёл, засранец?

Вареник в общих чертах объяснил, что в перерыве между работой "--- разливанием соков, перетаскиванием гастр и превращением продуктов в различного рода геометрические тела "--- он был занят исключительно перекусами, и для супа места просто не осталось.
Жену объяснение не устроило.

"--*Да не с кем мне там изменять! "--- развёл руками Вареник.
"--- Ты ж помнишь то кафе, там работают одни пончики!

"--*Весьма милые пончики, между прочим!
С сахарной пудрой!

"--*Ну да, милые.

"--*Ясно.
Развод и девичья фамилия.

Периодически Вареник и Вероника играли в ревность.
Но без души, без чувства.
После нескольких лет совместных скитаний, безденежья, ссор с родственниками и друзьями настоящая ревность кажется чем-то ребяческим и глупым.
Примерно как новомодные диеты.

\section{Коллектив}

Коллектив попался хороший.
Наверное, потому что сытый.
Люди очень часто несчастны лишь потому, что плохо питаются.

"--*Всему научим, бля, "--- грузно ворковала повариха Маша.
"--- На подхвате, нахуй, поработаешь "--- и на раздачу.
Ебало у тебя презентабельное, язык подвесим.
Народ здесь заебись, чётенький.
Только академовские, блядь, часто залетают, тут пересадка недалеко.
Сложно с ними, Вареник, ох ебать сложно\ldotst
Я блядь нахуй ваще порой не въезжаю, как с ними перетирать.

Раздача пошла бодренько.
Вареник внезапно обнаружил в себе незаурядный талант к общению с людьми.

"--*Идиоты! "--- орали с раздачи поварам.

"--*Внутри каждого идиота скрывается личность! "--- неожиданно глубокомысленно парировали с кухни.

Вареник немедленно нанёс изречение на лист бумаги и повесил на вытяжку, прямо над своим рабочим местом "--- разумеется, в зоне видимости только себя.
С тех пор оно помогало ему общаться даже с самыми неприятными клиентами.
Впрочем, внезапно прибывшее начальство распорядилось снять пакостную бумажку "--- видимо, не оценило глубину мысли.
Или оценило.

\spacing

"--*Жратва "--- это, конечно, здорово, "--- скептически сказал Шенкерман.
"--- Но что там кроме этого?
Карьерный рост, деньги?

"--*Я рад и тому, что есть, "--- ответил Вареник.

Трудностей работа почти не вызывала.

"--*Серьёзно, почти как на складе, только без тяжестей.
Единственная поебота с паспортами качества "--- там регулярно проходят проверки.
Приходится писать на завтрашнее число с вечера.

"--*Паспорта с завтрашним числом?
На продукты?

"--*Ну да, "--- ухмыльнулся Вареник.
"--- А что?

"--*Да ничего, "--- буркнул Шенкерман.
"--- Просто ты как-то чересчур легко говоришь о том, что являешься соучастником преступления.

Вареник поперхнулся и сменил тему.

\section{Жена}

Существует забавная теория о громоотводах.
У человечества есть громоотводы ненависти, к коим, безусловно, любил относить себя Шенкерман "--- всякий раз, когда его называли <<жидопидарасом>>, что случалось явно чаще честных президенстких выборов в России.
Вареник скептически относился к словам приятеля, но в существование громоотвода грусти поверил бы без лишних слов.
Одним из них была Вероника, и судя по объёму грусти, который она успешно утилизировала, Земля не утопала в печали благодаря не более чем десяти столпам.

Вероника работала в <<Векторе>>, новосибирском центре вирусологии и биотехнологии.
Вареник мало знал о том, чем именно занимается жена;
она предпочитала не распространяться.
Гораздо больше Вероника рассказывала о коллегах "--- у кого какие проблемы, кто как живёт.
По-видимому, тихая и чуткая девушка служила на предприятии нештатным психологом "--- ей рассказывали всё, от подробностей рождения до самых страшных личных тайн.
Иногда после очередного сеанса психотерапии Вероника приходила домой хмурая, и Вареник знал "--- столп Земли нужно поддержать.

"--*Опять передозировка? "--- шутил он, обнимая жену перед сном.

Вероника поглубже зарывалась носом мужу в подмышку и молча засыпала.
Наутро всё как рукой снимало;
а вот Вареник мучился кошмарами целую неделю и мысленно благодарил небеса, что эту странную роль "--- роль громоотвода "--- Судьба назначила не ему.

\section{Благотворительность}

"--*Солнце, "--- позвала Вероника.
"--- Ты занят?
Тут из <<Незнайки>> новое задание пришло.

"--*Где моя футболка с Суперменом?! "--- тут же отпустил дежурную шутку Вареник.

Вареник очень любил благотворительность.
Ему давно уже хотелось завести детей, но регулярно возникали какие-то проблемы "--- наподобие недавнего увольнения.
Объявления о сборе средств на лечение детишек печалили его, а порой и приводили в ярость.

"--*Сволочи ёбаные, мошенники, "--- плевался он, увидев очередной пост ВКонтакте.

Но год назад ему удалось-таки найти в Новосибирске настоящую благотворительность.
Малоизвестный фонд <<Незнайка>>, основанный энтузиастами из Академгородка, оказался достаточно прозрачным и надёжным "--- в частности, он никогда не собирал пожертвования деньгами.
Интересы у фонда были достаточно широкие "--- он помогал детским домам, ветеранам войн, инвалидам и многодетным семьям.

Вероника относилась к затеям мужа достаточно скептически "--- до первых заданий.
Вскоре супруги стали постоянными волонтёрами фонда.

"--*Ну что там? "--- спросил Вареник, поцеловав жену в щёку.

"--*Всероссийская благотворительная акция <<Цветы для бабушки>>.

"--*Для кого?

"--*Для бабушки, зай.
Читай.

Вареник пробежал глазами электронное письмо.

"--*Ясно.
Футболка отменяется, старушки юмор не оценят.

"--*Берём?

"--*Ну конечно.

\section{Сидр}

"--*Название какое-то странное, "--- критически заметила Вероника, вытаскивая лифчик через рукав футболки.
"--- <<Цветы для бабушки>>.
Как будто всё, что сейчас нужно бабушкам "--- это цветы.

"--*Я сам так же подумал.
И не я один.
Я сегодня звонил в <<Незнайку>>.
Волонтёры предлагали доставить продукты, починить технику, всякое такое.
Организаторы акции наотрез отказались.
Цветы "--- и точка.
<<Запросы составлялись с учётом мнения бабушек, бабушкам от вас ничего, кроме внимания, не нужно>>.

"--*И бабушки странные, "--- продолжала Вероника.
"--- Нетипичные.
Обычно они так и норовят что-то ухватить, нужное-ненужное "--- плевать.

"--*Может, эти совсем старые, при царе выращенные, "--- пошутил Вареник.
"--- Ладно.
Машину я заправил.
Ты отпросилась на завтра?

"--*Ага.
Давай заодно после бабушек в Икею завернём.
Нужно полочки купить, коврик и полотенца.

"--*И сидр.

Вероника расплылась в улыбке.

"--*И сидр.
Грушёвый.

"--*Может, я тебе всё-таки нормальный куплю, с алкоголем?

"--*Нет.
И не смей при мне называть ненормальным икеевский сидр.

\section{Дорога}

Вероника уже давно спала, откинув голову на сиденье и открыв рот.

"--*Слушай, вот зачем тебе это понадобилось, а? "--- сонным голосом спросил Шенкерман.
"--- Я не выспался\ldotst

"--*А я тебя предупреждал, что завтра едем.
Цветы не помни.

"--*Ладно, ладно!

Уже на трассе Вареник вспомнил, что забыл посмотреть адрес.

"--*Так, третий поворот должен быть, там дальше по картам посмотрим.

"--*Я телефон не взял, "--- пожаловался Шенкерман.

"--*Молодец, подготовился.

Вареник вытащил свой.

<<Критическое обновление.
Пожалуйста, подождите, установка может занять\ldotst>>

"--*Сука, да вы издеваетесь.

"--*Что ещё?

"--*Обновление, мать его.
Вообще нужно спрашивать мнение пользователя, удобно ли ему\ldotst

"--*Да кому всралось твоё мнение?
У Гугла бизнес-план горит.
Надо же поскорее напихать в твой кирпич ещё больше трекеров, рекламы и прочего говна.
Гугл "--- он как моя мамаша придурошная: понимает с полуслова, но лезет куда не просят!

Вареник громко выругался и забросил телефон в бардачок.

"--*Нику будить?

"--*Добудишься её сейчас.
Ника!
Солнце ты моё!
Телефон твой нужен!

Ответом был короткий грудной всхрап.

"--*Мда.

"--*Давай я у неё в кармане пошарю, "--- предложил Шенкерман.

"--*Так, блядь, никакой акробатики у меня в машине!
Тем более на трассе!

"--*Ну а хуле делать тогда?

"--*Хуле делать?
По старинке будем действовать "--- останавливаться и спрашивать людей.

"--*Кого мы встретим в восемь часов утра, в воскресенье?

"--*Таких же дебилов, как и мы, Володька, только с работающими, блядь, телефонами.

"--*Я надеялся, что ты скажешь <<Повернём назад и поедем досыпать>>.

"--*Не дождёшься.
Букеты уже купили, блин.
Поздно уже, сука.

"--*Какая гадость эта ваша благотворительность.

"--*Заткнись, а?

Вареник дёрнул коробку передач и раздражённо вдавил педаль газа.

\section{Атлас}

"--*Вареник, "--- осторожно начал Шенкерман.

"--*Что, Володя? "--- Вареник говорил преувеличенно-ласково;
руки сжимали руль до характерного хруста.
Это был восьмой круг по дорогам Зеленообска в поисках прохожих.

"--*Я тут в бардачке нашёл карты автодорог.
Семьдесят пятого года.
Кажется, я знаю, куда ехать.

"--*Отцовский, "--- хмыкнул Вареник, бросил взгляд на карту и вывернул руль.
"--- Он всегда ездил по этому атласу.
Мудрый был человек.

\section{Туалет}

Вдруг Вареник понял, что ему срочно нужно в туалет.

"--*Извините, "--- понизил он голос, "--- мне бы это\ldotst

"--*Пошли, "--- улыбнулась старушка.
"--- Так, тебе бумажка же нужна\ldotsq

Она ушла в другую комнату и вернулась с кипой пожелтевших от времени бумаг.

"--*Держи.
Сортир снаружи, у того зелёного здания.

"--*Вот спасибо, "--- обрадовался Вареник и махнул Шенкерману, чтобы тот подождал.

\section{Дорога обратно}

"--*Это что? "--- поинтересовалась Вероника, показывая бумаги.

"--*Это мне бабушки дали, чтобы я в туалет сходил, "--- объяснил Вареник.

"--*Ну ничего себе, "--- возмутилась Вероника.
"--- Это чей-то дневник был, ты в курсе?
Ты подтёрся чужим дневником, гад.

"--*Ну извините, рассматривать у меня времени не было, мысли были заняты другим! "--- съязвил Вареник.
"--- Что за дневник-то хоть?

"--*<<Костомарова З.\,П., вторая Восточно-Саянская экспедиция, 1929 год>>, "--- Вероника заулыбалась и пролистала дневник.
"--- Ух ты, тут <<яти>> машинописные.
Дореволюционная грамматика.
Впервые вижу машинописные <<яти>>, такое вообще в природе существует?

"--*Ну ясно.

"--*Что тебе ясно-то, солнце? "--- раздражённо буркнула Вероника.
"--- Дневник из экспедиции, почти столетней давности.
По-твоему, нормально этим жопу подтирать?

"--*Они мне его сами дали.

"--*И?

"--*Давай я съезжу в следующую субботу, верну его и извинюсь, хорошо? "--- устало сказал Вареник.
"--- Я не хочу снова ехать через эту пробку.

"--*Вот так бы сразу.
Но вначале я его прочитаю.

"--*Договорились.

\section{Несуществующий город}

"--*А прикол, ля, в том, что города этого, ля, ни на картах нету, ни в справочниках! "--- Шенкерман развёл руками.

"--*Ты уверен, Володь?

"--*Зайди в гугльмапс, ля!
Какой в опу Зеленообск?
На этом месте лес и голая трасса, ни поворота, ни города!

"--*Напиши в техподдержку.
Не может быть.
Помнишь атлас, по которому мы ехали?
Семьдесят пятого года, да?

"--*Сейчас два ноль двенадцатый, ля.
И гуглю я доверяю больше, ля, чем твоим совковым картам.
Те малость устарели, ля.

"--*Но ты же сам мне его нашёл на карте, Володька!

"--*Мало ли что я нашёл спросонья, "--- буркнул Шенкерман.
"--- Не мог целый город, ля, вот так взять и испариться.
Значит, либо мы были в другом месте, либо\ldotst на этом мысль останавливается.
Ну сам подумай, ля, "--- мы у кого-то спросили название города?
Нет.
Мы в диком недосыпе приехали утром, зашли к бабушке, подарили цветы.
Всё, ля.

"--*А адрес?
Улица Фрунзе, дом\ldotst

"--*Вареник, не тупи.
Улица Фрунзе, ля, бывает везде.

"--*Да, точно, сорян.
А почему все были в курсе, что мы приедем?

"--*А откуда я знаю, ля?
Акция всероссийская!

Вареник был вынужден признать, что друг прав.
Мистики ноль.
Они просто ранним утречком приехали неизвестно куда и подарили цветы неизвестно кому.

Однако, приехав домой, Вареник раз за разом в мыслях возвращался к душераздирающему пейзажу, который встретил их в Зеленообске.
Дома-<<пустышки>>, поросшие бурьяном бетонные плиты на дорогах, отсутствие неоновых вывесок и вообще какого-либо освещения.
Отсутствие прохожих, наконец.

Подобное запустение было чем-то из ряда вон выходящим, даже для России.
Этот город мог существовать на Курилах, в якутской тайге или ином Богом забытом месте.
Но не в центре Новосибирской области.

Впрочем, возложенная фондом миссия была успешно выполнена, и новоявленный Сайлент-Хилл забылся за чередой обычных забот.

\section{Банкомат}

Вероника была в реанимации уже четвёртый день.
Позвонила Татьяна, реанимационная медсестра и близкая подруга Вероники;
сообщила, что пока состояние стабильное.

\emph{Пока.}

<<Солнце, только не оставляй меня\ldotst>> "--- про себя взмолился Вареник, утирая набежавшие слёзы.

На готовку сил уже не было.
Вареник попросил у продавщицы какие-то котлеты и гарнир.
У продавщицы сил тоже не было.
Она с кислой рожей ходила туда-сюда, наполняя баночки едой.
Вареник вдруг вспомнил, как пытался подрабатывать в ресторане, надеясь на халявные харчи;
однако ресторан "--- это не столовая.
Запах и вид \emph{не его} пищи отбивали аппетит на всю смену и до глубокой ночи.

<<Триста пятьдесят, "--- обречённо считал Вареник, слушая писк кассового аппарата.
"--- А то и все четыреста>>.

Едва дождавшись радостного <<Покупка одобрена>>, Вареник выдернул карту, словно боялся, что жадное устройство высосет с неё все деньги.
Затем отправился к банкомату.
Банкомат принял карту, что-то промурлыкал завлекательным женским голосом, а затем задумался над надписью <<Подождите, операция выполняется>>.
Минута, две\ldotst без изменений.

У Вареника не нашлось сил даже грустно вздохнуть.

<<Блеск.
Прекрасное окончание прекрасного дня>>.

Cancel. Cancel.
Тот же результат.
Вареник на автомате вытащил телефон и набрал номер техподдержки банка.
В голове медленно проплывали мысли "--- о просроченной плате за съёмную квартиру, о задолженности по кредиту.
Ему очень хотелось сесть ничком прямо здесь и горько заплакать, но он стоял с телефоном у уха и бесстрастно слушал всё тот же мурлыкающий, завлекательный голос, говорящий о возможностях и перечисляющий цифры, которые следует нажать.

Банкомат вдруг ожил и выплюнул карту.

<<Сука>>.

Вареник подхватил карту и, едва подавив желание прописать банкомату хай-кик, отправился к соседнему.
К счастью, на этом неприятности в тот день и закончились.

\section{Врачебная интуиция}

Снова раздался звонок от медсестры Татьяны.
Татьяна не любила Вареника и никогда не называла его по имени;
для неё он был <<этот, как его>>.
Ещё до того, как Вареник и Вероника поженились, она всячески настраивала подругу против жениха.
Татьяна ни разу не была у них дома, даже на праздники.

На этот раз что-то изменилось.

"--*Алло, Валера, "--- раздался в трубке грубый, прокуренный женский голос, "--- Ника в порядке, её переводят из БРИТа в терапию.

"--*Что с ней, Тань?

"--*Диагноза нет, "--- по голосу Татьяны было понятно, что она очень встревожена.
"--- Тут скандал небольшой у нас был.
Реаниматолог молодой, Серёга, только универ закончил полгода назад\ldotst
В общем, он ей переливание плазмы сделал.

"--*И?!

"--*И ей помогло, температура спала до нормы, она вышла из комы и быстро идёт на поправку.
Он сам толком объяснить не может зачем, мямлит что-то про несуразное\ldotst
Мы с Палсанычем просмотрели \emph{всё}, даже её детскую карту подняли "--- никаких показаний, типичная клиническая картина тяжёлой инфекции.
Единственное возможное объяснение "--- что бы это ни было, в плазме были антитела.
Серёге, конечно, втык сделали, но, учитывая\ldotst

"--*Я ничего не понимаю, "--- признался Вареник.

"--*Валер, мы сами в шоке и ничего не понимаем.
В общем, приезжай сегодня с двух до четырёх, она в шестой палате, я тебя у входа встречу.

\section{Мультиверсум}

Шенкерман тупо хлопал глазами.
Строчки кода сливались в цветастую картину: оператор цикла трудолюбиво крутил мельницу, скрипя итератором;
оператор выбора придирчиво осматривал переменные, как меняла монеты;
красиво курил, ожидая своей очереди, красавец \verb|return|.
А вот и один из багов, указанных в репорте.
Шенкерман почти видел, как эта жирная сволочь оплела код, смяв и раздавив щупальцами хрупкие перекрытия вычислений.
Аналитическая геометрия.
Статистика.
Вероятность.

Большинство считает, что монетка падает лишь одной стороной вверх.
Математики знают вероятность выпадения одной из сторон.
Улыбчивый калека Стивен Хокинг вообще утверждает, что монетка падает и орлом, и решкой "--- в параллельных реальностях.

У Шенкермана было ощущение, что орёл и решка из параллельных реальностей встречаются друг с другом "--- или даже с самими собой.
Иногда он встречал самого себя, бросившего универ или оставшегося с той самой девушкой.
Иногда люди, с которыми он так и не познакомился, почему-то оборачивались ему вслед.
А у его вдовы при встрече с ним ёкало сердце, хотя она даже не знала его имени.

Шенкерман любил теорию мультивселенной так же, как и теорию вероятности.
Простор для фантазии.

<<Интересно, какова вероятность того, что меня на этой неделе уволят?>> "--- промелькнула странная мысль.

<<Единица>>, "--- ехидно сказал чей-то голос.

<<Да пошёл ты>>, "--- обиделся Шенкерман и снова занялся кодом.

\section{Архивы}

% Переводчик в дореволюционную орфографию - http://slavenica.com/

\begin{verbatim}
17-ГО IЮЛЯ, ГОДА 1929.
Пришелъ день, когда тяйшхаэры обѣщали посвятить меня въ тайну долго-
лѣтія. Сайры-хаэръ привелъ меня въ юрту старѣйшинъ. Онъ просилъ,
чтобы я протянула руку, сдѣлалъ надрѣзъ наконечникомъ стрѣлы и вы-
пилъ немного моей крови. Предполагаю,что это был некiй обрядъ - пле-
мя стало относиться ко мнѣ едва ли не со священнымъ трепетомъ послѣ.
/полное описаніе обряда - смотри въ приложеніи IУ /

26-ГО АВГУСТА, ГОДА 1929.
Я ощущаю нѣкіе измѣненія самочувствія. Я теряю въ вѣсѣ, и эта болѣз-
ненная худоба напоминаетъ мнѣ дѣвическую. Лицо поблѣднѣло, пропали
морщины, глаза стали большими, ихъ цвѣтъ - болѣе яркимъ. Мнѣ хочется
крови /зчрк/ кро /зчрк/ Я подозвала Бердышку, сдѣлала ему небольшой
порѣзъ, но это лишь распалило жажду. Мнѣ очень страшно. /исключить
эту и послѣдующія записи до 28/IX-29г. включительно изъ рукописи для
университета/
\end{verbatim}

Сердце Вареника заколотилось.
Он лихорадочно листал страницы пожелтевшего дневника.

\begin{verbatim}
29-ГО АВГУСТА, ГОДА 1929.
Сайры-хаэръ - замѣчательный человѣкъ. Онъ объяснилъ, что нужна чело-
вѣческая кровь. Да, это звучитъ ужасно, но ея нужна небольшая капля.
Я разсказала тт. о произошедшемъ, и Тоня дала мнѣ каплю своей крови.
Ихъ поразила произошедшая во мнѣ перемѣна. Онѣ сказали, что тоже хо-
тятъ вѣчной молодости. Но Сайры-хаэръ предупредилъ, что вѣчную моло-
дость мы сможемъ поддерживать, только если рядомъ будетъ одинъ "ПРО-
СТОЙ ЧЕЛОВѢКЪ" /вѣроятно, тотъ, кто не прошелъ обряда "ПОСВЯЩЕНІЯ"/.
Того, кто отказываетъ себѣ въ крови, ждетъ долгая и мучительная ги-
бель отъ жажды.

6-ГО СЕНТЯБРЯ, ГОДА 1929.
Тоня и Надя, не посовѣтовавшись со мной, получили "ПОСВЯЩЕНІЕ". Онѣ
радуются обрѣтенной молодости и, забывъ о цѣли экспедиціи... Я не
могу писать, я болѣю даже отъ мысли объ этомъ. Если узнаютъ въ уни-
верситетѣ, ихъ ждутъ весьма серьезныя проблемы. Надежду къ тому же
ждетъ мужъ и дочь. Я ругала ихъ вчера, но увѣщеванія на нихъ не дѣй-
ствуютъ.

13-ГО СЕНТЯБРЯ, ГОДА 1929.
У Тони пропалъ рубецъ на животѣ. Надежда сняла очки и сказала, что
ей они болѣе не требуются. Моральное разложенiе продолжается. Что
дѣлать? Какъ мы вернемся обратно? Надю это не безпокоитъ. Она сказа-
ла, что ея мужъ и революцію вытерпѣлъ, и войну, и каплю крови для
нея не пожалѣетъ.Чортовы дуры! Онѣ не понимаютъ, чѣмъ это обернется!
Зараза, которой мы по доброй волѣ заразили себя, опаснѣе любой конт-
ры. Она раздѣляетъ людей на тѣлесномъ уровнѣ. Если позволимъ ей рас-
пространиться - люди раздѣлятся, какъ тяйшхаэры, на не знающіх болѣ-
зней "ВЫСШИХЪ" и "ПРОСТЫХЪ", служащіх лишь источникомъ здоровой кро-
ви. Но тт. всё равно, онѣ думаютъ лишь о себѣ. Надежда готовится по-
святить въ тайну свою дочь.

20-ГО СЕНТЯБРЯ, ГОДА 1929.
Сегодня я впервые увидѣла свое лицо въ зеркалѣ. На видъ мнѣ не болѣе
16-ти лѣтъ. Придется покинуть Москву, въ противномъ случаѣ будетъ
слишкомъ много вопросовъ. Т. Кормилинъ получитъ мою рукопись, но я
сомнѣваюсь, что онъ сможетъ и далѣе скрывать мое происхожденіе.
Къ Бабинымъ въ Томскъ? На свой страхъ и рискъ навѣстить проѣздомъ
Костомаровыхъ въ Новониколаевске? Мнѣ страшно. Я не смогу заснуть
ночью.

28-ГО СЕНТЯБРЯ, ГОДА 1929.
Я наконецъ-то смогла достучаться до Тони и Нади. Думаю, онѣ осознали
опасность, нависшую надъ нами.Мы въ срочномъ порядкѣ пересматриваемъ
полевые дневники, чтобы исключить всѣ матеріалы о послѣдствіяхъ "ПО-
СВЯЩЕНІЯ". Надежда не разстается съ замысломъ "ПОСВЯТИТЬ" свою дочь.
У меня въ сумкѣ лежитъ Маузеръ, который отдалъ мнѣ т. Кормилинъ. Бо-
юсь, мнѣ придется имъ воспользоваться.
\end{verbatim}

\section{Помощь}

Вокруг цвели деревья, жужжали насекомые, пели птицы, но Шенкерман видел молчаливый, холодный и белый покров зимы.

<<Как же люди одиноки в этом мире, "--- думал Шенкерман.
"--- Вся жизнь напоминает обучение нейросетей.
Место рождения, родители, тело, общество "--- случайные исходные данные, непонятные сигналы, с которыми что-то нужно сделать, чтобы доказать свою пригодность>>.

<<Тебе ли жаловаться на семью! "--- загремел в голове голос матери.
"--- Другие руки распускают, а то и вовсе нищие алкаши!
Мы тебе и образование дали, и одели-накормили\ldotse>>

Далее шёл длинный список заслуг матери перед Шенкерманом.

<<И всё-таки я от вас сбежал, "--- грустно подытожил Шенкерман.
"--- Плоть от плоти вашей, плод воспитания вашего.
Продолжайте винить дьявола и плохую компанию>>.

Шенкерман вдруг с какой-то нежностью вспомнил Вареника.
Почти сразу раздался звонок, вернув Шенкермана в жестокую реальность "--- к работе, пустой квартире и ноющей шее.

"--*Да.

"--*Володя, мне нужна твоя помощь.

"--*Вареник, у меня послезавтра дедлайн по проекту.
Оставить его мне не на кого.

"--*Но\ldotst

"--*Слушай, меня это всё уже заебало.
Благотворительность-хуительность, бабки эти ваши странные\ldotst
У меня работа, от которой зависит, буду я кушать или нет.
Всё прочее "--- не более чем хобби.
Отъебись хотя бы на неделю.

"--*А жизнь Вероники "--- тоже хобби?

Шенкерман поперхнулся.

"--*Что?

"--*Я нашёл старые записи об экспедиции на Саяны.
Скорее всего, она подхватила редкий вирус.
Вернее, её заразили.

"--*А я тебе кто "--- вирусолог?
Вези её в больницу!

"--*Нельзя.

Шенкерман судорожно вздохнул.
Вареник плакал в трубку.

"--*Блядь, Вареник!
Если меня уволят, ты труп, тебе ясно?
Буду через двадцать минут.

\section{Следственный эксперимент}

"--*Вареник, ты понимаешь, что я не специалист?

"--*А куда обратиться? "--- взорвался Вареник.
"--- В больницу?
К Нике на работу?
Может, сразу в госбезопасность?

"--*Если то, что написано в дневнике "--- правда, "--- задумчиво сказал Шенкерман, "--- то лучше молчать.

"--*Я тебе о чём и говорю!

"--*Слушай, "--- Шенкерман потёр нос, "--- если я понял тебя правильно, в данный момент жизни Ники ничто не угрожает.

"--*Да.
Но мы не знаем\ldotst

"--* \ldotst надолго ли и что будет дальше, "--- закончил Шенкерман.
В его глазах блестели мысли, сменяя друг друга, как кадры кинофильма.
"--- Я понял.
Говоришь, дневник тебе дали на подтирку?
Значит, хозяйка сама не знала о его значении.

"--*Либо это была не хозяйка, "--- буркнул Вареник.
"--- Ты ж видел, у них там всё общее, они друг к другу постоянно в гости ходят.

"--*Бабки однозначно в этом замешаны, "--- уверенно сказал Шенкерман.
"--- Они жили там десятилетиями, а потом пришла Ника и бац "--- заразилась.
Так не бывает.
Заражённых определённо больше.
Я попробую поискать в открытых источниках, а тебе, думаю, стоит съездить к бабкам.

"--*Как?
Ты же сказал\ldotst

"--*Третий поворот на трассе, а потом по атласу семьдесят пятого года.
В прошлый раз это сработало.
Если в деле не замешаны зелёные человечки, Великие Учителя или барабашка с планеты Нибиру "--- сработает и в этот.
Давай-давай, Вареник, пошёл, ты щас моё рабочее время тратишь.

\section{Странные смерти}

Вареник зевнул и потёр глаза.

"--*Я долго спал?
Ты меня не разбудил.

"--*Долго.
Я пытался.

"--*Сорян.

"--*Не извиняйся, я компенсировал всё твоей едой на день.
Вероника вкусно готовит.

"--*Да ради Бога, мне не жалко.
Нашёл что-нибудь?

"--*Нашёл кое-какие статьи.
Пару месяцев назад прошла череда странных смертей.
Все по одной и той же схеме "--- человек бледнеет, худеет, начинает бредить, затем впадает в ярость, затем теряет сознание "--- и смерть.
Первоначально у них не наблюдалось признаков инфекции, поэтому умирали они, как ты понимаешь\ldotst

"--* \ldotst в дурке, "--- закончил Вареник.
"--- Что за статьи?

"--*Одна статья из новосибирской газеты, одна запись на барабинском паблике ВКонтакте, ещё один похожий случай в Новокузнецке сняли на видео.
Видео называется <<ВАМПИР НАПАДАЕТ НА ЛЮДЕЙ!!!>>

"--*Вампир?

"--*Ну, тот, кто снимал, почему-то решил, что этот псих хочет выпить крови.
А, вот.
Ещё психиатр из Новосибирска на Задолба.ли, практически повторяет историю из газеты.
Сразу несколько незнакомых друг с другом людей начали бросаться на прохожих с разницей в пару дней.
Он предположил, что в городе появился какой-то новый синтетический наркотик.

"--*Сможешь узнать, в какую больницу их повезли?

"--*В третью.
Я уже там был днём и даже выяснил, кто, скорее всего, написал на Задолба.ли.
Но поговорить не выйдет.

"--*Почему?

"--*Он поссорился с женой, выпил, видимо, ну и схватил кровоизлияние в мозг.

"--*Умер?!

"--*Выжил.
Но лучше бы умер.
В общем, я сказал, что веду расследование, оставил им там свои контакты.
Если будут похожие случаи, со мной свяжутся.

\section{Поцелуй Иуды}

"--*Что она сделала, вспоминай.

Вероника наморщила лоб.

"--*Просто причесала меня, и всё.

"--*Расчёска была какая?

"--*Большой такой старинный гребень.

"--*Железный?
С острыми зубьями?

"--*Да.
Настолько острыми, что\ldotst

"--*Она тебя поцарапала, "--- перебил Вареник.

"--*Откуда ты знаешь?

Вареник промолчал.
Его била дрожь.

"--*Совсем чуть-чуть, зай, "--- пробормотала Вероника.
"--- У самого лба.
Долго извинялась, а потом поцеловала.

"--*Поцеловала?

"--*Да, в лобик.

"--*В лобик, где ранка?

"--*Ну да!
Что в этом такого?

Вареник вместо ответа наклонился к Шенкерману и взлохматил его волосы, показав царапину.
Затем ткнул пальцем в свой лоб.

Вероника ахнула.

"--*А знаешь, в чём разница? "--- сказал Шенкерман.
"--- Нас не целовали.

\section{Увольнение}

"--*Вам есть что сказать?

Шенкерман молчал.
Можно было придумать тысячу причин, почему проект не был готов.
Перебои со связью, обновления стороннего ПО.
Но бессонная ночь высосала из него все силы.

"--*Жена друга заболела, "--- честно ответил Шенкерман.

"--*Насколько я помню, вы разработчик, а не медик.

"--*И тем не менее понадобилась моя помощь.

Алексей Анатольевич вздохнул.

"--*Клиент потребовал заплатить неустойку в размере четырёхсот тысяч рублей, угрожая оглаской.
Плюс небольшая сумма за время простоя серверов.
Как Вы понимаете, для SibDeepTec'а это ощутимые финансовые и репутационные потери.
Согласно ТК РФ, мы не имеем права взыскивать с Вас более месячного оклада.
Тем не менее, учитывая, что Вы полностью признаёте свою вину, мы рассчитываем на то, что Вы добровольно погасите хотя бы половину долга.
В этом случае мы дадим Вам возможность написать заявление по собственному желанию и избежать проблем с трудоустройством в Новосибирске.
Если же вы, кхм, погасите весь долг, мы готовы оставить Вас при должности и ограничиться выговором.

Шенкерман улыбнулся.
<<Умирать, так с честью>>.

"--*Алексей Анатольевич, не ранее чем месяц назад я просил, чтобы мне в помощь дали двух стажёров.
Но Вы решили сэкономить.
Мои, цитирую, <<профессиональные навыки вполне достаточны, чтобы работа была выполнена в срок>>.

"--*Сожалею, что оценка Ваших профессиональных способностей была завышенной.
Но\ldotst

"--*А вчера я не мог до вас дозвониться ни по одному из трёх номеров, чтобы сообщить, что у меня появились неотложные дела.
Можно поинтересоваться, где вы были в преддверии дедлайна?
Может, вы тоже подрабатываете врачом в свободное время?

"--*Не хамите, Шенкерман.
В данный момент мы говорили\ldots

"--* \ldotst об ошибке, "--- перебил Шенкерман.
"--- Именно о Вашей ошибке как руководителя, не принявшего меры на случай непредвиденных обстоятельств.
Увольняйте по статье.
Мой остаток за месяц пойдёт на уплату неустойки, остальное "--- только через суд.
Трудовую можете оставить на память.

Шенкерман повернулся и вышел вон, хлопнув дверью.

\section{Дорога домой}

Домой в Бердск Шенкерман ехал, глядя на пылающий закат в окне полупустой маршрутки.
Закат время от времени вырывался из плена высотных зданий, мостов, придорожных кафе и густого переплетения леса;
каждый его поцелуй вызывал у Шенкермана лёгкую улыбку.
Напротив сидела девушка, и закат заставлял искриться оранжевым её прямые каштановые волосы.
Лица Шенкерман так и не увидел.
Или просто не запомнил.

"--*Володь, "--- голос Вареника был едва различим в трубке, "--- мы с Никой на первоначальный взнос копили, деньги есть.
Мы и так уже начали потихоньку ковырять сбережения.
Возьми пятьдесят штук.

Шенкерман отказался.
Не смог.

Звонил директор SibDeepTec'а.
Угрожал подать в суд.

"--*Идите на хуй, "--- сказал ему Шенкерман и сбросил звонок.
Он знал, что до суда дело не дойдёт, и был прав.

Чтобы успокоиться, он написал ВКонтакте Саше Красноруцкой, с которой общался в универе.
Кажется, они познакомились на чьём-то дне рождения\ldotst нет, на рок-концерте.
Точно на рок-концерте.
Она тогда перебрала со спиртным, и он тащил деваху с розовой чёлкой до общаги, с трудом отбив её у каких-то сомнительных бритоголовых парней.
Шенкерман хорошо помнил, как дрожали его ноги "--- и отнюдь не из-за веса безвольного женского тела.
После окончания юрфака Саша состригла чёлку, вышла замуж за горнолыжного инструктора и уехала с мужем на Кубань.

<<Они пытаются тебя развести, дарлинг, "--- подтвердила Саша.
"--- Подать в суд "--- это как на всю страну заявить о том, что у тебя в компании творится.
Не ссы, не нужен им такой гемор из-за полуляма опилок.
В принципе, ты у них даже сам можешь отжать энную сумму, если есть желание>>.

Желания у Шенкермана не было.

<<Хуй с ними, согласна.
Короче, будут проблемы "--- вот мой номер, звони сразу, у меня систер в прокуратуре по Дзержинскому району.
Если надо "--- сама приеду>>.

<<Добро иногда возвращается>>.

Шенкерман вдруг понял, что жутко хочет спать.

\section{Есть контакт}

Повинуясь какому-то внезапному порыву, Вареник вывернул руль и поехал в ночь.

Трасса.
Дальний свет выхватывал поднявшуюся пыль, ночных мошек, испуганный силуэт какого-то крупного грызуна\ldotst
Вот и нужный поворот.

Вареник выключил фары и как можно тише подогнал машину к зелёному зданию.
Затем, заглушив двигатель, пешком отправился в сторону общаг.

Окна общаги светились слабым неверным светом, какой издаёт свеча или старая керосиновая лампа.

Вареник не стал второй раз искать способ забраться в общагу.
Он вернулся в машину и стал ждать рассвета.

Наконец, едва солнце вступило в свои права, он постучал в запертую дверь общаги.
Грохот разнёсся по тихой пустынной улице;
Вареник был уверен, что стук услышали во всех окрестных домах.

Вареник выждал минуту и постучал ещё "--- на этот раз гораздо громче.

Дверь распахнулась.

"--*Чё надо? "--- осведомилась бомжеватого вида старуха, тыча в Вареника обрезком трубы.
"--- Вали на хуй, здесь наше место!

"--*Мне нужна Костомарова З.\,П., "--- сказал Вареник.

"--*Не знаю такую, "--- буркнула бабка.
"--- Вали, кому сказано!

Вареник помахал у неё перед носом удостоверением кладовщика.
Киря как-то сказал, что оно очень похоже на милицейское.
Мужики поверили ему на слово "--- он состоял на учёте с двенадцати лет.

"--*Отведите меня к тому, кто знает Костомарову, или у вас будут проблемы, "--- отчеканил Вареник.

Бабка замолчала и выпрямилась.
Бомжеватость исчезла без следа.

"--*Только быстро, "--- сказала она.

\section{З. П.}

<<Права была Ника, "--- промелькнуло у него в голове.
"--- Нетипичные какие-то старушки.
Я никогда не видел матёрых вояк, но эти как будто прошли от Москвы до Берлина, причём не санитарками.
А я "--- как немец на допросе\ldotst>>

Старушки действительно стояли словно гвардейцы по команде <<вольно>> "--- расслабленно и в то же время соблюдая определённый порядок.
У одной из них за поясом красовался мясницкий нож.

Кресло было повёрнуто спинкой к двери "--- так, что над ней были видны только пара седых волос, искрившихся на солнце.
Вареник уже хотел подойти ближе, чтобы взглянуть на лицо "--- и наткнулся на острый, как штык, взгляд старушки-гвардейца с ножом.
Этого делать явно не стоило.

"--*Мне сказали, что вы знакомы с Костомаровой З.\,П., "--- начал Вареник.

"--*Да, знакома, "--- раздался из кресла чистый, совсем не старческий голос.

"--*Кем вы ей приходитесь?

"--*Для начала представьтесь.

"--*Валерий.

"--*Отчество, фамилия, звание.

"--*Валерий Николаевич Сафронов, тысяча девятьсот восемьдесят пятого года рождения, рядовой запаса.

"--*Не нужно ёрничать, товарищ милиционер.

"--*Я не из милиции и не собираюсь устраивать вам допрос.

"--*Мне сказали, что вы предъявили удостоверение, "--- прохладно сказал голос.
Вареник краем глаза заметил, что старушка с ножом словно невзначай потянулась к оружию.

"--*Мне пришлось соврать, "--- быстро сказал Вареник.
"--- Я здесь не из пустого любопытства и не из научного интереса.
От вашей информации зависит моя жизнь.

Наступило молчание.
Старушки вопросительно посмотрели на спинку кресла, ожидая слова хозяйки.

"--*Что вам нужно? "--- наконец спросило кресло.

"--*Мне хотелось бы поговорить с вами о дневнике Костомаровой, который она вела во время экспедиции в Тыву.

Голос хрустально усмехнулся.

"--*Глах, так вот кто мой дневник-то стащил.
Ну-ка разверни меня, погляжу на этого молодца.

Старушка с ножом подошла и легко, словно пушинку, развернула кресло к Варенику.
Тот, не отрываясь, смотрел на сидящую в нём женщину.

"--*Это вы "--- Костомарова З.\,П.?

"--*Зинаида Павловна, к вашим услугам, "--- улыбнулась она.
"--- Не можешь поверить, что мне за сотню?

"--*Кое-что выдаёт возраст, "--- осторожно сказал Вареник.
"--- Ноги.

Женщина улыбнулась белыми, как волчьи клыки, зубами.

"--*Если бы Глаха носила что-то поприличнее этой стариковской хламиды, ты бы увидел, что её ноги утрут нос любой молодой.
А я сломала спину и тридцать лет как прикована к кровати.
Могу только руками шевелить, и то немного.
Увы, от таких случайностей молодость не защищает.

Зинаида Павловна кивнула на пустое кресло.

"--*Садись, чего стоишь-мнёшься.
Я хоть и из дворян, да дворянства давно уж нет.
Вижу "--- дело серьёзное, раз меня искать решил.

\section{Начало войны}

"--*Сюда уже едут, "--- сказала Зинаида Павловна.
"--- Они не должны никого найти.
Никого живого и тёплого.

"--*И что ты предлагаешь, Зинка? "--- осведомилась одна из женщин.

"--*Я предлагаю вам добровольно вернуть долг за долгую здоровую жизнь.

Поднялся гвалт.

"--*Да иди ты в пизду, полоумная!

"--*Она права!

"--*Да где права-то?
Бежать надо!

"--*Если кто-то хочет бежать "--- собирайте вещи, "--- сказала Зинаида Павловна.
"--- Я вам больше не командир.

Толпа разделилась;
больше двух третей ушли.

"--*Отлично.
Теперь окончательно выяснили, кто из нас порядочный человек, а кто нет.

"--*Так что делать-то, Зин? "--- спросила Глафира.

"--*Да известно что.
Не хотят долги возвращать "--- будем взыскивать.
Глаш, ты у нас фронтовая, можешь девкам объяснить, что да как?

Глафира обратилась к остальным:

"--*Нина, Маша "--- доставайте своё из загашника, смазать не забудьте.
Остальные "--- найдите ломы да ножи понадёжнее.
Начнём с этой общаги, с нижнего этажа.
Вы двое "--- на пожарную лестницу у входа, вы четверо "--- на заднюю.
Пока все на позициях не будут "--- молчим и глазами не сверкаем.
Всем ясно?

Все дружно кивнули.

"--*Сафронов, "--- в глазах Глафиры блестел огонь бойца, "--- ты машину где оставил?

"--*Возле зелёного здания, во дворике.

"--*Придут как пить дать с шоссе, там уже их техника на обочине.
Для тебя этой дороги больше нет.
Быстро отведи машину в лесок, её не должны увидеть.
Жди в старой лесницкой, пока тебя не позовут.
Если до темноты никто не придёт "--- поезжай без оглядки дальше в лес, по сухой земле куда-нибудь да выберешься.
Нина, покажи ему.

"--*Идём со мной, хлопец, "--- старушка с пугающе чёрными глазами подхватила Вареника под локоть.
"--- Да живей, не стой!

\section{Война}

В старой лесницкой царил полумрак.
Половые доски всегда скрипели, строение словно жило своей собственной жизнью.
На столе стояла проржавевшая до дыр посуда;
в полутёмном углу, под самым потолком, примостились поблёкшие образа.

Вареник старался не смотреть в сторону икон.
Он знал, что невиновен, что он не сделал ничего плохого;
но глаза Иешуа, которых время почти не коснулось, смотрели пугающе обвинительно.

Снаружи наступили сумерки.
Вскоре в лесницкую окутала кромешная тьма, но Вареник так и не решился даже вытащить телефон, чтобы проверить время.

Вдруг снаружи раздался скрип и звук шагов.
В голове Вареника замелькали мысли:

<<Машину я ветками забросал, в ложбине\ldotst но что если\ldotsq меня найдут по номерам, что сказать\ldotsq может\ldotsq>>

Дверь избушки со скрипом распахнулась, и у Вареника замерло сердце.

"--*Ты тута, хлопец? "--- ворчливо спросил женский голос.
"--- Да тута, я ж чую тебе.
Выдохни, всё вже хорошо.

Снова раздался скрип, и в избушку не без труда въехала тележка.

"--*Нина, переложи меня на кровать, "--- попросил голос Зинаиды Павловны.
"--- Неудобная эта карета, неудобная\ldotst

Чиркнула спичка, и в углу под иконами одна за другой загорелись три лампадки в красном стекле.

Нина легко, словно пёрышко, сняла с тележки обтянутый тканью скелет Зинаиды Павловны.
Скелет сверкнул молодыми зубами в сторону Вареника.

"--*Проголодался, наверное, тут сидючи?

"--*Ну проголодаться не проголодался, а вот в туалет сходил, когда вы в дверь начали ломиться, "--- просипел Вареник.
Руки всё ещё немилосердно тряслись.

"--*Подвигайся, хлопец, "--- буркнула Нина.
"--- Зинку на кровати положу\ldotst

\section{Норна}

Вареник обернулся и увидел Глафиру.
От ворчливой старушки не осталось ничего "--- седые волосы лежали на плечах, словно у древней норны, голубые глаза горели неистовым огнём фанатика-берсерка.
Её рубаха и лицо были в крови, словно она разрывала тела врагов голыми руками.
Войдя в лесницкую, Глафира чинно повернулась к образам и трижды осенила себя крестным знаменем.

"--*Готово, Зин, "--- хмуро сказала она, вытирая нож.
"--- Тебя\ldotsq

"--*Не нужно, "--- остановила подругу Зинаида Петровна.
"--- Прошу тебя, возьми ещё грех на душу "--- помоги товаркам и себе.

"--*Да какой грех, Господи, "--- сказала Глафира.
"--- Коли не стану душегубкой, так Иисус Христос, Сын Божий, с меня вчетверо спросит "--- за то, что позволила жить лукавому.
Идём, девочки.

"--*Не, Глаха, я сама, "--- проскрипела Нина.
"--- Рука у тебя верная, что ни говори, но за себя сама отвечу, не обессудь.

\chapter{Неразобранное}

\section{Кризис среднего возраста}

Вареник отчасти понимал, откуда растут ноги у кризиса среднего возраста.
Всю молодость люди ставят себе какие-то цели.
Школьники должны закончить школу.
Студенты "--- закончить универ.
Выпускники "--- найти тёпленькое местечко по специальности.
Холостые должны жениться, а бездетные "--- завести ребёнка.
И вот, когда все цели достигнуты, в голове человека возникает закономерный вопрос "--- а что, собственно, дальше?
Этот вопрос, как контрольный выстрел, добивает окончательно человеческую психику, и без того расшатанную школьными порядками, бессонными ночами перед сессией, орущим дитём и <<работой над отношениями>>.

Отчасти Вареник даже был благодарен судьбе, что его <<победный марш>> оборвался на стадии поступления в универ.
Могло быть и хуже.

\section{Универ}

Универ, в котором отучилась Вероника, переживал не лучшие времена.
Раньше, говорят, все таблички на кабинетах были латунные, даже именные.
Сейчас же все они были бумажные, наскоро набранные на компьютере.
Имена и должности сменялись на них регулярно.

\section{Сигарета}

С прочими друзьями Вареник общался ровно до того момента, пока с ними было о чём покурить;
последняя сигарета истлела года два назад.

\section{Прозвище}

Своё не очень приличное прозвище Валерка получил ещё в школе;
он достаточно рано начал ухлёстывать за девочками и имел среди них популярность, и пацаны прозвали его Вареником.
Валерка на прозвище не обиделся и и вскоре сам стал представляться так.
Потом чрезмерное увлечение противоположным полом прошло, а погоняло осталось.
Все привыкли "--- Вареник и Вареник.

\section{Сирена}

В офисе то и дело срабатывала пожарная сигнализация, и её с матом или тяжёлыми вздохами отключали.

"--*Да когда ж её, блядь, отрегулируют! "--- громко выражал общее мнение завхоз.

Время от времени где-то вдалеке выла сирена, похожая на сигнал воздушной тревоги.
Выла она так же громко и противно, но отрегулировать её ни у кого желания не возникало.

Сигнал тревоги для России "--- обычное дело.
Когда звучит сирена, люди закатывают глаза и говорят <<Опять>> "--- и это в лучшем случае.
Сигнал тревоги "--- это не повод тревожиться, это часть навязанного Западом стереотипа о безопасности труда.
В России опасностей на производстве нет и быть не может "--- во всяком случае, пока на шахте не погибнут горняки или в кинотеатре не сгорит сотня детей.
Впрочем, с горняками все уже тоже как-то свыклись.
Никто же их туда не гнал, верно?
Сами же шли работать?

\section{Полис}

Полис Шенкерман сделал в универе.
На втором курсе пришлось лечь в больницу с травмой, и внезапно выяснилось, что его полис недействителен.
За новым пришлось ехать аж в Бердск "--- очередные чудеса российской логистики.

Едва он нашёл офис страховой компании, на него набросилась работница:

"--*Молодой человек, вы компьютерщик?

"--*А что, так заметно? "--- буркнул Шенкерман.

"--*У нас интернет пропал.

Пока Шенкерман разбирался со слетевшими настройками интернет-соединения, ему сделали полис.
Работница выпроводила помощника с тысячей благодарностей, всучила ему солидно выглядящий документ и огромную шоколадку с коньяком.

Только на остановке Шенкерман понял, что с него не взяли денег.

\section{Ум}

Он представлял собой самый нелюбимый работодателями типаж "--- живой и изобретательный ум, который совершенно бесполезен в коммерческом плане.

"--*Если ты такой умный, почему ты такой бедный? "--- спрашивали его друзья.

"--*Это к Богу, "--- отвечал он.
Друзья смеялись и возвращались к своим иллюзиям.

\section{Грех}

"--*Знаешь, во времена Гражданской был один такой комиссар, из обедневших дворян\ldotst
Прапрадед мой со стороны отца.
В тридцатые его из-за происхождения в расход пустили по накатанной, но делов натворить он успел.
Особенно любил духовенство троллить.
Поймает очередного попа, поставит у стенки и задаёт контрольный вопрос "--- куда вошло копьё Лонгина Сотника?
И если ответит поп "--- в сердце, комиссар ему тут же пулю в печень.
А всё потому, что читать Писание бездумно, не зная анатомии "--- грех.

"--*А если поп отвечал правильно?

"--*В голову стрелял.
Дворянин всё-таки был, не пролетарий голожопый.

\section{Ночь}

Любимым временем Шенкермана были четыре часа утра летом.

В пять часов дня на улицах гуляет куча приличного народа с детьми.
В восемь исчезают дети.
В одиннадцать исчезают приличные люди.
В час исчезают гопники с бьющимися бутылками и <<туц-туц-туц>>.
В два ложатся двинутые соседи, в три идут спать ночные гонщики, и, наконец, в четыре ты остаёшься со Вселенной наедине.

\section{Постоянное}

Шенкерман зашёл помыться и с омерзением оглядел общую ванну.
Жившая у соседей старуха, очевидно, подмывалась и не убрала за собой.
Ванну Шенкерман помыл, но плавать в растворе чужого дерьма, пусть и гомеопатическом, ему не хотелось.
Он наскоро принял душ, стараясь стоять на самом конце ванны, который едва заливала вода.

"--*Слушай, "--- спросил он у Вареника, "--- ты не думаешь переезжать?

"--*Куда?

"--*Неважно куда.
Ты мне недавно жаловался, что детей всё не завести "--- то одно, то другое, то третье.

"--*И что?

"--*Дети "--- это последнее, о чём можно мечтать в такой хибаре.
В конце концов, работа у тебя сейчас есть, поищи вариант получше.

"--*Это был временный вариант после того, как нас с Никой выкинули на мороз, "--- извиняющимся тоном сказал Вареник.

"--*И вы уже полгода здесь, "--- подытожил Шенкерман.
"--- Нет ничего более постоянного, чем временное.

\section{Никита}

Взять какого-нибудь человека.
У него есть план, как и у миллионов других "--- школа, университет, работа, дети.
Он совершенно обычный "--- вежливый, в костюмчике, занимается тем же, чем все, отдыхает так же, как все.
Дела расписаны чуть ли не по дням и часам.

Но в один момент он делает что-то не так.
Например, после универа в родной стране он бросает девушку и едет в холодную Сибирь получать второе высшее.
И в этот момент, когда он принял решение, обратной дороги нет "--- его судьба обречена быть не такой, как у других.
Потому что к нему начинают относиться, как к особенному человеку, он начинает притягивать особенных людей.

Иногда кажется, что его жизнь катится ко всем чертям, но это только прибавляет ей индивидуальности.
Этот человек будет жить по-другому, он будет счастлив, он будет искать свой путь окольными тропами.
И однажды, возможно, он протопчет дорогу специально для себя.

А люди, которые жили по шаблону, будут несчастны по тому же шаблону.
Потому что счастья по шаблону не бывает.

\section{Кошка и мышь}

"--*Загадка: как найти чёрную кошку и чёрную мышь в тёмной комнате?

"--*И как?

"--*По хрусту.

"--*Не понял.

"--*Скоро поймёшь.

\section{Звезда по имени}

"--*Ну не знаю.
Я всё ещё верю, что где-то там надо мной светит моя счастливая звезда.
Осталось только её найти.

"--*Эта звезда светит всем, и искать её не надо.
Она называется Солнце.

\section{Взросление}

"--*Я иногда помечтать любил, "--- сказал Шенкерман.
"--- В шараге особенно.
Знаешь, воображал себя мегакрутым воином или что-то подобное, когда типа раздеваешься до пояса и идёшь ломать руки-ноги.
И вот однажды мечтаю я себе.
Зима, снегопад, толпа народу.
И выходит против меня такой громила.
Я таких больше всего боялся "--- кого свалить у меня тупо силёнок не хватит.
Раздевается громила до пояса, мышцами играет, насмехается надо мной.
А я, вместо того, чтобы тоже раздеться, надеваю шапку и заправляю свитер в штаны.
Зима же, снегопад.

"--*И к чему это?

"--*Да так, "--- ухмыльнулся Шенкерман.
"--- Я просто в тот момент понял, что повзрослел.

\section{Карьера}

Своей карьерой Вареник был обязан порезанному пальцу.
В тот знаменательный день он окончательно поругался с матерью, и та отказалась высылать ему деньги.
Вареник не особо расстроился.
Он расстроился бы, если бы у него были на это душевные силы "--- поиск работы шёл не ахти как, деньги заканчивались, да и ссора с матерью не была первой в его жизни.
<<Заебала>>, "--- вздохнул Вареник и пошёл мыться.

Палец резануло неожиданно.
Вареник всегда гордился своей реакцией;
падающая зубная щётка не успела коснуться пола.
Однако стеклянная полочка имела своё видение ситуации.
Палец залил кровью все поверхности, словно обезглавленный неумелым палачом преступник;
на срезе проступили сухожилия "--- к счастью, целые.

Вареник собирался после душа поискать работу, но палец окончательно его доконал.
Он залил рану какой-то дрянью, попытался свести воедино края кожи и перемотал злосчастный палец бинтом;
затем, чтобы успокоиться, без особых надежд занялся старым проектом, который безуспешно пытался раскрутить последние два года.

Проект взлетел в тот же день.
Финансовые проблемы на время ушли в прошлое.

\section{Призрак}

Есть нечто прикольное в том, чтобы любить призрака.
Призрак никогда тебя не разочарует.
Призрак не исчерпает себя как личность, ведь он "--- твоё продолжение с лицом другого человека, и глубина призрака равна твоей глубине.
К призраку не нужно стремиться.
А в самые тяжёлые моменты можно с безобидным эгоизмом мечтать, что есть кто-то, кто без тебя никогда не будет счастлив, кто всегда тебя знал и ждал.
И эта фраза "--- глупая, на самом деле "--- всегда будет на самом краю той не требующей реализации мечты: <<Где ты, сука, всё это время был?>>

У Вареника такой призрак имелся.
Даже сейчас, после пяти лет в браке по большой любви, он иногда доставал телефон и долго смотрел на фото десятилетней девочки, игравшей на пианино.
Вероника думала, что это его троюродная племянница.
На самом деле племяннице было уже под тридцатник, и жила она в далёком Берлине с мужем и детьми.
Вареника не мучила совесть.
Он считал, что у каждого человека есть то сокровенное, о котором не стыдно врать "--- даже самым близким людям.

Никто, кроме тебя, не имеет прав на твоего призрака.

\section{Матримониальные планы}

Тётя Изольда была единственной роднёй в Новосибирске, которую он признавал;
несмотря на это, порой она умудрялась выесть ему мозг без соли и специй.

"--*К нам в гости придут тетя Лиля и дядя Саша Фельдман с дочкой Лиорой.
Милая девочка, черноглазая, кудри по пояс, медицинский заканчивает в будущем году!
Я вас познакомлю, ма ихпат лахэм?

"--*Бабушка, я к вам обязательно приеду, но пожалуйста, давайте обойдёмся без матримониальных схем!
У меня есть девушка, не еврейка, но я её люблю.
Кристина, я про неё рассказывал.

"--*Пока я её не увидела, у тебя нет девушки! "--- крупные зубы тёти Изольды сияли даже в телефонном разговоре.
"--- Шучу, не волнуйся.
Понимаю, сейчас времена другие, не то что раньше\ldotst
В четверг ха-Ацерет, приезжай в пятницу, дома будет только Яша, я что-нибудь тебе приготовлю.
Я знаю, что ты атеист, просто хочу тебя повидать.

Насчёт девушки Шенкерман соврал.
Тётя терпимо относилась к атеизму, но отсутствие у внучатого племянника личной жизни приводило её в исступление.
Если бы не придуманная девушка, знакомство с Лиорой Фельдман произошло бы в добровольно-принудительном порядке.
Прецеденты были.

\section{Встреча выпускников}

"--*Ну ничё ты так красавец, "--- ухмыльнулся Дубынин, пожимая Шенкерману руку.
"--- Слышал, ты щас серьёзный человечек, при деньгах.
А помнишь в школе?

Шенкерман помнил.
В школе Дубынин не особенно стеснялся в выражениях, если речь заходила о нём.
Выражения сопровождались отвратительным гоготом компании пацанов.
Этот звук всегда издаёт компания и никогда "--- один человек.
Этот звук всегда издают при свидетелях и никогда "--- без.
Этот звук издают те, кто совершает нечто неприглядное и уверен в своей безнаказанности.
Шенкерман считал, что этот отвратительный звук нужно запретить законодательно и ввести в отношении нарушителей самые суровые санкции.

"--*Я мало что помню со школы, "--- сказал Шенкерман.
"--- В конце концов, это старая история.
Девчонок интересуют не старые истории, а сегодняшний счёт в банке, верно?

"--*Девчонок всегда интересуют только мужики, Шенкерман, "--- захохотал Дубынин.
"--- Вонючие брутальные мужики с хуем.

"--*Ну есть такие, которых можно купить за вонючий хуй, "--- пожал плечами Шенкерман.
"--- Только на бабки они ещё резвее клюют.

\spacing

Шенкерман не мог найти себе места.
Вначале он прибился к компании Вареника, но, поняв, что не хочет слушать эту чушь, двинулся дальше.

Два Артёма.
Хороший, добрый парень по кличке Немец и второй, так и не получивший за время школы погоняло.
Рассказы про недавнюю годовщину свадьбы, футбол, работу\ldotst
Дочка Немца недавно попала в больницу.
Диагноз "--- расстройство пищевого поведения.

"--*Как я её понимаю, Немец, "--- прокомментировал второй Артём.
"--- У меня тоже расстройство одно, как вспомню, сколько денег на еду уходит.
Иногда в безденежье даже кусок в горло не лезет, одни цифры в голове\ldotst

Шенкерман, почувствовав неприятное желание закрыть руками лицо, поспешил уйти и от них.

Ильгиза.
Улыбчивая быстроглазая татарка "--- с ней Шенкерман танцевал на выпускном.
Ещё в школе обладала нюхом на мужиков и вполне неплохо устроилась.
Правда, насчёт источников заработка её мужа у Шенкермана были подозрения.
Для сетевика он зарабатывал чересчур много и стабильно.
Да и мало кто из сетевиков держит дома две единицы длинноствола.
Ильгиза, как и приличествовало её уровню интеллекта, тактично о них умолчала;
но Шенкерман успел заметить на фотографиях её дочери характерные продолговатые сейфы.

Вскоре к ним подошёл Дубынин и стал прислушиваться к разговору.
Его присутствие немного нервировало.

"--* \ldotst ну и вот, я ему предлагала, такая говорю, у тебя всё на мази, научи меня, чё как, вместе зашибать будем, тем более что Коська щас уже большая.
А он такой "--- да зачем тебе это надо, вон лучше полимеркой занимайся, клиентуру набирай.
Так что реально на <<Колхиде>> подняться, реально.

"--*Я ни разу не слышал, чтобы хоть кто-то выбивался в люди с помощью <<Колхиды>>, "--- сказал Шенкерман.
"--- Гораздо больше знаю тех, кто прогорел.
Ну или тех, кто крутится как белка в колесе прямо с девяностых, со времени основания, ищет клиентов без продыху\ldotst

"--*Ну меня муж зарабатывает на этом! "--- развела руками Ильгиза.
"--- Я тебе уж пиздеть не стану!
Я не говорю, что он на жопе ровно сидит "--- пропадает порой, конечно, на неделю-две, но лавэ мне на карту идут как по часам!
Не из воздуха же он их берёт, верно?

"--*Ой, да хуй с вами, "--- махнул рукой Шенкерман.
Он знал, что прав, но спорить не хотелось.

Дубынин внезапно набычился и подошёл к Шенкерману вплотную.

"--*Ты меня щас на хуй послал, уёбок? "--- прошипел он.

"--*Я не с тобой разговаривал, вообще-то, "--- опешил Шенкерман.
От неприятного предчувствия в животе затрещал металлический храповик.

"--*Очки сними.

"--*Что?

"--*Очки сними, сука.
Не хочешь?
Как хочешь.

Шенкерман слишком поздно сообразил, что это была игра в благородство, и пропустил мощный удар кулаком в глаз.
Металлическая оправа очков с противным хрустом чиркнула по кости, из рассечённой брови хлынула кровь.
Мир засиял и завертелся.

"--*Ах ты гнида, "--- пропыхтел Шенкерман и встал, цепляясь за стену.
Прошла целая вечность, прежде чем он вскинул руки в жалкой пародии на боксёрскую позицию "--- как учил в детстве добрый, но так и не ставший родным человеком отец.

"--*В позу встал? "--- рассмеялся Дубынин.
"--- Помахаться хочешь?
Ну давай, хуле.

В Шенкермана прилетел ещё десяток ударов руками и ногами.
Каждый удар впечатывал его спиной в стену.
Но он почти их не чувствовал.
В крохотном зазоре между вскинутыми руками Шенкерман видел горящее ненавистью лицо Дубынина.

<<Один удар, пальцами по глазам или кулаком в нос, один раз вытянуть руку\ldotst
Давай, Володя, ты же работал как-то в универе со стадвадцатикилограммовой, набитой песком грушей\ldotse >> "--- завывал голос в голове.

<<Не смей, "--- твердил другой.
"--- У тебя недостаточно практики и скорости.
Вытянешь руку "--- снова получишь в лицо>>.

"--*Дуб, ты охерел, что ли? "--- заорал третий голос "--- на этот раз по другую сторону от гудящих черепных костей Шенкермана.
Вареник схватил Дубынина за грудки и оттащил от заваливающегося друга.

"--*Дуб, блядь, ты ёбу дал?
Остынь нахуй!

"--*Не, а чё он, бля? "--- Дубынин всё ещё делал символические попытки прорваться к Шенкерману.
"--- Чё он, сука, блядь, начал?

"--*Мне нахуй одному надо эту тушу держать? "--- заорал Вареник.
"--- Он бухой в говнище, когда успел только.
Толик, Стас, ну ёб вашу мать!

Вскоре Вареник уже сидел возле друга и критически осматривал рассечённую бровь.

"--*Ну нихуя, блядь.
Чё случилось-то?

"--*А я откуда знаю? "--- пробормотал Шенкерман.
"--- Я вообще с Ильгизой разговаривал\ldotst

"--*Слышь, Шиша! "--- заорал откуда-то из-за мужских спин Дубынин.
"--- Ты блядь можешь хоть все лавэ в Новосибе заработать.
Я помню, каким ты был, Шиша, помню!
Нихуя не изменилось!
Нихуя!

"--*Пошли-ка отсюда, "--- сказал Вареник и, подняв с пола погнутые, заляпанные кровью очки, осмотрел их со всех сторон.
"--- Заебись встреча выпускников, блядь.
Давай, хватайся за меня, щас такси поймаем\ldotst

\section{Ужин}

"--*Да и так ясно, к чему всё идёт, дорогой! "--- перебила его тётя Изольда.
"--- Думаешь, я слепая и глухая?
Вчера чеченцы, сегодня "--- евреи!
Плавали, знаем!

"--*Бабушка, я не думаю, что евреев будут убивать.

"--*Вот когда синагоги начнут сжигать, тогда и вспомнишь мои слова.
Владимир, в Израиле живёт Лина, она тебе поможет.
У нас вообще там родни ещё с советских времён\ldotse
Если я слово скажу "--- пол-Тель-Авива встанет!

Шенкерман вымученно улыбнулся.
Ему очень не хотелось разрушать иллюзии тёти Изольды.

"--*Я уже давно не общался с Линой.

"--*Какая разница?
Она "--- твоя сестра!
Вот что.
Я сама ей напишу.
И не смей говорить <<нет>>, напишу и всё.

"--*Не посмею.

"--*Вот и отлично.
Давай, дорогой, кушай.
Тут ещё осталось два кусочка пирога.
Я не верю, что тебе он не понравился.

\section{Несвершившийся факт}

"--*В дневнике было написано, что у вас был <<маузер>>, "--- сказал Вареник.
"--- Вы\ldotst вы всё-таки решились?

"--*Нет, не решилась.
Решись я тогда "--- не было бы всех девочек.
Это Надя <<посвятила>> дочь, подруг, а дальше покатилось как снежный ком.
Скольких трудов мне стоило их собрать вместе\ldotst

"--*А что случилось с Кормилиным? "--- спросил Вареник.

"--*А, ты про того, который, "--- лицо Зинаиды Павловны погрустнело.
"--- Я уж и забыла про него, как давно всё было\ldotst
Репрессировали его в тридцать первом.
Но меня он не выдал.
Любил он меня, видишь ли.
Замуж всё предлагал.
Что с ним дальше стало "--- не знаю.

"--*Ваш труд так и не увидел свет?

Женщина покачала головой.

\section{Судьба племени}

"--*Тяйшхаэры были совершенно особым племенем.
Многие историки мечтают найти что-то особенное, непохожее на других, а мне повезло.
На самом деле они были потомками финно-пермских племён, ближе к саамам, непонятно как оказавшихся в Саянах.
Это подтверждалось многим "--- легендами, ритуалами, языком, который походил на удивительно красивую смесь монгольских и финно-угорских корней.

"--*Что произошло потом? "--- спросил Вареник.
"--- Куда делись тяйшхаэры?

Зинаида Павловна помолчала.

"--*Через год после моего отъезда в Урянхае начались волнения.
К тяйшхаэрам пришли болезни.
Тиф выкосил почти всех <<непосвящённых>>, оставшиеся заболели сифой.
<<Посвящённые>> решили уйти в степи, чтобы выжить, но столкнулись с карательным отрядом чекистов.
Тяйшхаэры не поняли, что за железные палки в руках у их противника, и приготовились защищаться.
Мне это потом рассказали тувинцы, которые всё видели.
Они говорили, что с гор спустились могучие богатыри с тугими луками, одетые в богатые одежды из шкур и перьев\ldotst

Зинаида Павловна заплакала.

"--*Ты понимаешь, хороший мой, у тяйшхаэров была совсем другая одежда, не как у аратов.
Они ехали верхом на быках, а не на лошадях.
Их язык напоминал тыва лишь отдалённо.
Но чекисты этого не знали.

\section{Слава}

Вот она, истинная слава белого человека.
Неважно, на каком языке он говорит "--- на английском, русском, испанском или французском.
Чукчи были гордым вольным народом, а стали героями анекдотов.
Тяйшхаэрам досталось забвение.

\section{Кровь}

"--*Можешь дать мне крови? "--- жалобно спросила Вероника.

"--*Ника, опять?
Да что с тобой?

"--*Мне очень плохо, милый.

В этот раз Вареник почти не колебался.

"--*Я наберу в шприц и дам тебе.
Согласна?

"--*Да, согласна!

Вареник вытащил из аптечки завалявшийся пятикубовый, сорвал шуршащую обёртку, затем зубами затянул жгут на плече.
Вероника жадными глазами следила за каждым его движением.

Едва кровь коснулась её языка, как странный огонь в глазах угас без следа.
Вероника обессиленно рухнула на диван.

"--*Со мной что-то происходит.

Вероника подняла руку и посмотрела на неё.

"--*Рука стала тонкой.
Пропали шрамы на запястьях.
Лифчик стал больше на один размер.

Вареник схватил руку жены, осмотрел её и застыл в ужасе.
Шрамов действительно не было.
Более того, на ладонях почти пропали линии.
Кожа была гладкой и эластичной, словно у топ-модели.

"--*Милый, что со мной?

В глазах Вероники замерла растерянность.

\section{Фирдеус}

Ещё в тот день не пришла таджичка, которая обычно протирала столы.

С таджичкой Шенкермана связывали самые чистые и светлые чувства.
Он обычно ставил в стол тарелку с печеньем, чтобы подкрепляться по ходу работы.
И однажды он начал замечать, что печенье пропадает "--- понемногу, но заметно для намётанного глаза.

<<Мне не жалко>>, "--- подумал Шенкерман и начал покупать больше печенья, чтобы пропажа казалась меньше.
Ему правда было не жалко.
Печенье стало убывать вдвое быстрее.
Наконец, когда у Шенкермана было особенно хорошее настроение, он оставил в тарелке записку со словами <<Приятного аппетита!>>.

Похитительница объявилась уже на следующий день.
Молодая чернобровая уборщица с ломаными извинениями поднесла Шенкерману целое блюдо традиционной выпечки.

С тех пор и началась их посменная "--- пять минут в день "--- пакистанская любовь.
Таджичка дочиста убирала стол, раскладывая бумаги и принадлежности именно так, как привык Шенкерман.
Он втихаря оставлял для неё сладости и всякие милые глупости "--- рисуночки, стихи, бумажные сердечки и цветные ленты.
Она украдкой рассматривала подарки в уголке, и её плоские смуглые губы приоткрывались в неуверенной, но белозубой восточной улыбке.

Однажды к Шенкерману подошёл завхоз.

"--*Ты сдурел? "--- тихо прошипел он.
"--- Она из диаспоры.
Если её родичи узнают, тебе пизда, однозначно.
Да и нахуя тебе сдалась эта чернавка, они не моются неделями!

Шенкерман вежливо перевёл тему.
Он с удовольствием предложил бы Фирдеус встречаться, если бы у той не было троих детей и больного подагрой торговца-мужа.
И это в её цветущие двадцать два года.

Печенье она таскала для детей в особенно голодный месяц.
Потом это, по её словам, превратилось в привычку.
Одна подработка, другая, третья\ldotst
Вся романтика, которую могла позволить себе эта женщина "--- цветные ленточки и бумажки от едва знакомого мужчины.

<<Какими же глупыми выглядят чувства на фоне такой жизни>>, "--- подумал Шенкерман.

Вечером он услышал краем уха, что Фирдеус уволилась.
Ни её фамилию, ни место, где она научилась так вкусно готовить, Шенкерман так и не узнал.

\section{4}

Лучше не стало.
Но стало легче.

\section{Коты}

Шенкерман любил котов, но ему с ними не везло.
Одного из них, темпераментного красивого тайца, он помнил особенно хорошо "--- от маленького прозрачнолапого котёнка до оскалившейся посмертной маски, над которой секунду спустя сомкнулась земля.
Но были и другие "--- менее красивые, менее темпераментные, но не менее любимые.
Серого пушистого бродягу он помнил только в один день.
Шенкерман тогда выпустил его на улицу, кот сходил в туалет и бодро понёсся за гаражи по каким-то своим кошачьим делам.

В этот день он пропал насовсем.
Кто-то из соседей, кажется, даже видел его труп;
но для Шенкермана серый кот так и остался гордым, сильным зверем, который весело скакал вдаль навстречу приключениям.
Последнее впечатление порой едва ли не важнее первого.

\section{Искренность}

Шенкерман любил Сашу.
Да, наверное, всё-таки любил.
Шансов у него не было никаких "--- её отношения были расписаны на десять парней вперёд.
<<Беги к мечте!
Прыгай через голову!
Подходи к недоступным девушкам!>> "--- этим пестрела лента ВКонтакте, заставляя его сверстников разбивать свои сердца и резать спины о чьи-то ножи.
Шенкерман умел реально оценить свои возможности и не пытался что-то предпринять.
Тем не менее, не проводить с ней время было выше его сил.

Как-то Шенкерман спросил Сашу:

"--*Скажи, как у тебя получается быть душой компании?

"--*Сэйм квесчен, "--- ухмыльнулась Саша.
"--- Полагаю, я просто искренняя.

"--*Искренняя? "--- удивился Шенкерман.
"--- И всё?

"--*Ну это так-то дохуя для нормального человека.
Мне повезло "--- я родилась отбитой наглухо.

"--*И как может помочь искренность?

"--*Душа компании "--- это уникальный чел, обладающий, как у нас на факультете любят говорить, \emph{харрризмой}.
Каждый чел по-своему уникален.
Но большинство лгут, пытаясь быть похожими на других.

"--*Ты хочешь сказать, что душой компании может быть каждый?

"--*Оф кос.
Надо лишь найти подходящую компанию.
А, ну или решить для себя, а надо ли тебе в принципе быть душой компании.
Это тоже хороший вопрос, потому что для незакомплексованного юнита плюшки сомнительные.

"--*Может, дело в характере?
Для искренности нужен характер.

"--*Так далеко в рассуждениях я не заходила, "--- призналась Саша.
"--- Передай пиво и хватит меня грузить.

\emph{Харрризма} стоила Саше дорого.
В её компании не было ни одного юриста, за исключением сестры;
она предпочитала тусоваться с гуманитариями и айтишниками.

\section{Жадность}

Нет, не прав был Иешуа Га-Ноцри.
Страшнейший из пороков "--- отнюдь не трусость.
Жадность Иуды до денег, жадность Синедриона до власти отправили Иешуа на крест, а отнюдь не старый, сломленный службой, одиночеством и болезнью прокуратор.
Жадность "--- вот самый страшный порок, и едва ли кто-то усомнится в этом, глядя на сегодняшнюю Россию "--- безмерно богатую и бесконечно нищую, словно профессиональный попрошайка.

\section{Цена человека}

Каждый человек имеет цену.
На фриланс-сайтах эта цена прописывается вполне конкретно "--- в долларах в час.
Человечество представляет собой гигансткий список работников, отранжированных по стоимости часа работы.
Буквально через пять лет после описываемых событий в Китае введут понятие социального рейтинга, но Шенкерман, узнай он это сегодня, даже не удивился бы.
Это вполне объяснимая попытка узаконить существующее положение вещей.
Впрочем, зачем вводить какие-то рейтинги, если человек и так имеет чётко прописанную цену?

Есть, правда, ещё один список людей "--- неофициальный.
В нём люди отранжированы по способности приносить убытки "--- за счёт собственных умений или социальных связей.
Те же люди, которые не способны приносить ни прибыль, ни убытки, справедливо считаются никчёмными.
Никого не волнует их судьба, никто не скорбит по поводу их смерти, кроме одного-двух значимых других "--- таких же никчёмных.

\section{Информатор}

"--*Алло, Владимир? "--- раздался в трубке взволнованный женский голос.

"--*Да, я вас слушаю.

"--*Это из третьей вас беспокоят, по поводу\ldotst ну вы в курсе.

"--*Я вас понял, "--- Шенкерман прижал плечом телефон к уху и подхватил блокнот.
"--- Вы хотите встретиться?

"--*Да.
Подъезжайте к <<Ауре>> к трём, встретимся у входа.

\razd

В час пик Большевичка напоминает засорившуюся трубу.
Маршрутка ползла со скоростью улитки.
Солнце припекало нещадно;
в маршрутке царил полумрак, наполненный раскалённым воздухом и тяжкими вздохами.
Не помогали даже открытые окна и люк "--- рядом рычал и пыхтел грузовик, поднимая клубы пыли и распространняя запах бензина.

Причина пробки вскоре обнаружилась.
Две аварии, разделённые двумястами метрами дорожного полотна.
Выжили все, но <<Матиз>> радостно загорал на крыше, вскинув в небо колёса и перегородив половину дороги.

<<Оказывается, если поставить его поперёк дороги, не такой уж он и маленький>>, "--- подумал Шенкерман.
Он вообще успел передумать все мысли, какие только были возможны в тот момент, выдумать парочку витиеватых матерных проклятий, когда вдали наконец-то показался стеклянный мост Речного вокзала.

Выйдя из метро, Шенкерман сразу влился в толпу народа, снующую возле Оперного.
Мимо проехал скейтер и, неудачно прыгнув, на секунду прилёг на постамент памятника Ленину.
Ленин же, как обычно, гордо поглядывал на светлое будущее в лице оживлённой автострады, явно собираясь что-то сказать спутникам.
Вооружённые ружьями солдаты, крестьянка с зазубренным мечом и рабочий, вскинувший жутковатого вида каменный топор, сурово-одухотворённо ждали слова вождя.

Кое-как выбравшись из толпы, Шенкерман дворами бросился к <<Ауре>>.
Он уже представлял мнущуюся у входа обеспокоенную женщину и был немало разочарован, увидев там только парочку скучающих секьюрити.

<<Так, а ведь уже три>>.
Шенкерман вытащил из кармана телефон и удостоверился в собственной правоте.

"--*Владимир Владимирович? "--- внезапно услышал он за спиной низкий мужской голос.

Шенкерман обернулся, и его сердце замерло.
Светло-голубая рубашка, сине-зелёный погон с двумя звёздами\ldotst

"--*Пройдёмте.

Шенкерман подчинился, даже не успев обдумать приказ.

\section{Пытки}

"--*Володьке с сердцем поплохело, он сейчас в реанимации, "--- сказал Вареник.
"--- Его в ментовке пытали, хотели выведать, что он знает про вирус из Зеленообска.
Если это правда, он тебя не выдал.

\section{Вероника}

"--*Я уволилась из <<Вектора>>, "--- пробормотала Вероника.
"--- Сказала, что устала и устроюсь в институте, поближе к дому и поспокойнее.

"--*Они не знают?

"--*Нет.

\section{Правила}

"--*Что это? "--- удивился Вареник.

"--*Правила.
Тебе их надо выучить и следовать им в точности.

"--*Это правила для семей, в которых живут ВИЧ-инфицированные.

"--*Именно.
Наш вирус определённо передаётся при контакте с кровью, а значит "--- правила те же самые.

Вареник пролистал брошюру.

"--*О, так мы можем\ldotsq

"--*Да, самец-засранец, мы можем.
Но только с презервативом.
Я разошлю кровь кое-каким людям, вместе попробуем выделить антитела к вирусу.
Затем я проверю его концентрацию во всех жидкостях организма.
Целоваться до той поры на всякий случай будем только так "--- чмок-чмок.

"--*А\ldotst

"--*Если захочешь пойти налево "--- я не возражаю.
Для меня сейчас важнее, чтобы ты был рядом.
Только не притащи в дом ещё один вирус, у нас и так перенаселение.

"--*Я искал тебя всю жизнь, "--- тихо сказал Вареник.
"--- Думаешь, я пойду налево сейчас?

"--*Вообще я не это имела в виду, но как хочешь "--- я тебя за язык не тянула, "--- уголки губ Вероники едва заметно приподнялись.
"--- Кстати, я попросила Таню, мне по её каналам сделают справку, что я ВИЧ-положительная.
И ты то же самое будешь говорить людям, если понадобится.

"--*Зачем?!

"--*Чтобы обезопасить окружающих.

"--*Ты головой думала, прежде чем просила?
Тебя могут уволить, от тебя друзья отвернутся!
Тебе ли не знать, как тяжело с этим жить!

"--*Настоящие друзья никуда не уйдут.
А то, что от меня будут шарахаться "--- это только плюс.
Вирус не должен уйти дальше этой тушки.

Вареник кивнул.

"--*Ты будешь изучать его?

"--*Только для того, чтобы обезопасить окружающих, "--- покачала головой Вероника.
"--- Он мне достаточно крови попортил.
Пусть сдохнет вместе со мной в безвестности.

\section{6}

Если за вечную жизнь приходится платить тем, что сам ты сотворить не в состоянии "--- это вечное рабство, а не вечная жизнь.

\section{Имя}

"--*Вареник, а как тебя всё-таки звать?

"--*Володь, ты издеваешься? "--- опешил собеседник.

"--*Я серьёзно, "--- Шенкерман выглядел слегка растерянным.
"--- Сколько тебя помню, ты всегда был Вареником.
А кто ты по паспорту-то?
Виталий?

"--*Валерий.
Валерий Николаевич.

"--*Ты потратил пять секунд на то, чтобы вспомнить, "--- заметил Шенкерман.
"--- Как прозвище-то въедается.

"--*Неправда.
Ты меня просто ошарашил вопросом.

"--*А давно тебя по имени-отчеству величали-то?

"--*А вот чёрт знает.
В политехе вроде.
У нас препод был смешной, всех звал <<мастерами>>.
Я был у него <<мастер Сафронов>> или <<Валерий Николаевич>>.

"--*Он тебе шанс давал, Вареник, "--- ухмыльнулся Шенкерман.

"--*Какой шанс?

"--*Недостаточно родиться Валерием Николаевичем.
Им надо стать.

"--*Ещё немного "--- и я стану антисемитом, "--- предупредил друга Вареник.

"--*Да хоть в скинхеды иди, "--- отмахнулся Шенкерман.
"--- Ты данные по вирусу принёс или мне гороскоп составлять для поиска?

\section{Шоколадка}

Во вторник зашла Саша Красноруцкая.
Она немного поправилась с тех пор, когда Шенкерман видел её в последний раз, и стала ещё красивее.
Наверное, родила.
Шенкерман не стал спрашивать.

Они говорили мало.
Впрочем, о чём тут говорить.
Саша молча смотрела на изукрашенное синяками тело Шенкермана, и по её точёным скулам переваливались желваки.

"--*Спасибо, "--- сказал Шенкерман.
"--- Я не ожидал, что ты прилетишь.

"--*Не за что, Володь, "--- ответила Саша.
"--- Повезло, что Оля про тебя узнала\ldotst

"--*Хорошо, наверное, быть юристом.
Всегда в курсе событий.

Саша лишь грустно усмехнулась.

"--*Так ты не расскажешь, под кого ты там копал?

"--*Мне нечего рассказывать.

"--*Выздоравливай, "--- Саша погладила Шенкермана по голове.
"--- И что бы ты ни делал, остановись сейчас.
Второй раз я тебя вытащить не смогу.

Сладости и фрукты, которые принесла бабушка, уже давно закончились.
За окном царила летняя ночь, наполненная шелестом листьев и песнями сверчков, но сна не было ни в одном глазу.

Шенкерман в приступе нелепой надежды потянулся к рюкзаку.
Иногда по рассеянности он забывал про оставленную в кармане шоколадку или пачку печенья;
когда пропажа находилась, это было всегда сродни маленькому чуду.

В этот раз чуда не произошло.

\section{Смысл без любви}

В среду приходил человек в штатском.
Был не особенно обходителен "--- в ответ на вполне законное возмущение лениво ткнул в лицо Петру Алексеевичу ксиву.
Травматолог, скрипнув зубами, кивнул и велел медсёстрам вывести соседей Шенкермана в коридор.
Говорил человек в штатском долго и пространно, обращался исключительно на <<ты>>.
Припомнил и живущую в Канаде бабушку, и сестру "--- гражданку Израиля, вскользь упомянул и про заказанные из Китая микроконтроллеры "--- <<статья сто тридцать восемь точка один у ка эр эф>>.
Смысл букв и цифр до Шенкермана дошёл: будешь жаловаться "--- будут проблемы.

"--*Может, всё-таки напишешь заявление? "--- спросил Вареник.

"--*Я устал, "--- сказал Шенкерман.
"--- Кто я, в конце концов?
Простой айтишник.
Ни связей, ни денег.

"--*Ты не простой айтишник, Володька.
Ты гений.
Ты фактически дело шестидесятилетней давности раскрыл, понимаешь?
Раскрыл и разрулил не по закону, а по совести!
Это тебе надо в органах работать, а не этому чекисту!

"--*Да кто меня пустит в органы? "--- горько усмехнулся Шенкерман.

Вареник понял.
Больше они эту тему не поднимали.

Шенкерман лежал и думал, что же не так с этой больницей?
Ответ пришёл постепенно "--- безвкусные стены были вымерены до сантиметра, по нормам СанПиНов;
над каждой розеткой висела столь же безвкусная табличка, сделанная исключительно из-за требований вышестоящих органов.
А кафель, который покрывал стену позади раковины, совершенно не сочетался по расцветке с основной стеной.
В каждой детали был смысл, но не было ни грамма любви.

<<Как и во всей моей грёбаной жизни, "--- грустно подытожил Шенкерман.
"--- К чёрту.
Разберусь с переломом "--- и на Родину\ldotst
Хватит с меня этой российской действительности>>.

\section{Конец}

"--*Извините, Зинаида Павловна, что так вышло.

"--*Не извиняйся, милый.
Надо было, конечно, сразу обо всём рассказать\ldotst
У нас ещё были идеалы "--- служить Родине, пролетариату\ldotst да вот знала я, что с нами сделают, коль проболтаемся.

"--*Вы думаете, вас бы убили?

"--*Хуже, милый.
У моей внучки, которая в Поволжье жила, как-то свекровь была.
Поехала в город на работу, а в родной деревне авария.
Сошёл с рельс поезд с какими-то химикатами, деревня вспыхнула, как спичка\ldotst
Неизвестно, остался ли кто в живых, но сверху велели согнать машины да запахали пепелище.
Приехала свекровь "--- а деревни нет.
Поехала узнавать в город "--- а её никогда и не было.
В лечебницу забрали и мытарили там пять лет\ldotst

Вареник почувствовал, что плачет.

"--*И войну братоубийственную запахали.
Муж мой сам пошёл.
Мог не идти, много уж ему лет-то было, да пошёл.
Написал как-то, что у немцев много солдат с дырками в спине.
От <<Вальтера>>.
Не трусы были, честные вояки.
А потом сам вернулся с пулей в спине "--- уже от ТТ.
Комиссар мне потом так и сказал: <<Скажи спасибо, красавица, что мы его тебе Героем Союза привезли.
Могли и по-другому>>.
А потом его победой "--- его Победой! "--- запахали сталинские грехи.
И Чернобыль бы с Припятью так же запахали, коль можно было бы.
Ты уж мне поверь.
Сейчас всё ж другие времена, и люди другие.
Сейчас не запашут.
Не позволят\ldotst

Зинаида Павловна закашлялась.
Вареник поднёс ей чашку полуостывшего чая.

"--*Идём, милый, я тебя причешу, "--- Зинаида Павловна улыбнулась мужчине вымученной улыбкой.
"--- Не бойся, просто причешу.
Всё одно сегодня умру\ldotst
Глашка говорит, что Богу душу отдают люди, да мы-то не такие, мы люди советские, атеисты.
Знаешь, какая школа-то у нас была?
Ууу\ldotst
Да иди, не бойся.

Вареник, поколебавшись, наклонился к ней.
Зинаина Павловна вытащила гребень "--- не железный, а обычный, деревянный.
Вареник закрыл глаза.
Он ощущал себя котом.
Как же всё-таки здорово, когда тебя чешут.

"--*Ну вот, "--- старушка отложила гребень в сторону и потрепала мужчину по щеке.
"--- Теперь иди.
Веронике привет от меня передавай.
Так и скажи ей "--- она последняя.
Будет благоразумной "--- последней и останется.

"--*Она уже хотела\ldotst

"--*Рано хотела.
Ну свели девки счёты с жизнью, ну и что с того?
Они-то пожили.
Вероника тоже пускай поживёт.
Всё одно, как ни крути "--- долгая молодость, здоровая жизнь, счастливая старость.
Всё одно дар, хоть и проклятый.
У вас ещё молодость-то прекраснее нашей "--- столько соблазнов, столько удовольствий, мир как на ладони.
Грех такую терять.

"--*А дети?

"--*Детей усыновите.
Много их нынче, сироток.
Кто, если не вы?

Вареник закинул сумку на плечо и проверил, на месте ли ключи от машины.

"--*Так значит, вы скоро умрёте? "--- обернулся Вареник к Зинаиде Павловне.
"--- А откуда вы это знаете?

Но она уже не дышала.

\section{Чемодан}

"--*Ну, в общем, это.
Я совсем без жилья.
Можно я у вас хотя бы переночую?

"--*Да оставайся, конечно.
Ты билет хотя бы сдал?

"--*На хуй билет.

"--*Ну как на хуй! "--- буркнул Вареник.
"--- Деньги, и немалые.

"--*Есть вещи важнее денег.

"--*Что ты наделал, Володь, "--- грустно сказала Вероника.
"--- Там у тебя родичи, другая жизнь.
А что здесь?

"--*Кризис не закончится, "--- поддержал её Вареник.
"--- Будет только хуже.
Ещё и конец света на Новый год обещают\ldotst

"--*Да не будет никакого конца света, Вареник, "--- грустно улыбнулся Шенкерман.
"--- Хуже "--- будет.
Но это не конец.

"--*Надеюсь, ты ещё передумаешь.

"--*Возможно.
Но только не сегодня.

Вареник и Вероника улыбнулись в ответ.

"--*В этом доме тебе всегда рады, "--- сказал Вареник.
"--- Оставайся, чё уж.
Может, ещё детей наших покрестишь.

"--*Я атеист, ты же знаешь, "--- осклабился Шенкерман.
"--- Обрезанный атеист.

"--*Но ты ведь не будешь отрицать, что между некоторыми людьми есть особая связь?

"--*Б-г здесь точно ни при чём.

"--*Возможно, "--- ухмыльнулся Вареник.
"--- Но получить его благословение не помешает.
В конце концов, на чьё ещё благословение мы можем рассчитывать в этом мире?

Шенкерман кивнул.

"--*Пойдём, я суп сварила, "--- улыбнулась Вероника.
"--- С тыквой.

"--*Люблю тыкву, "--- сказал Шенкерман и задвинул ручку чемодана.
Ручка громко щёлкнула.

\section{P. S.}

В 2029 году Вероника Сафронова-Зима была номинирована на Нобелевскую премию по биологии за открытие двух ранее неизвестных механизмов старения человека.
Тогда же она призналась, что она все эти годы являлась носителем редкого ретровируса и тайно изучала его свойства.
Валерий Сафронов развёлся с Вероникой в 2013 году и заключил фиктивный брак с гомосексуальной женщиной, чтобы его и её семья имели возможность усыновления детей из приюта.
Семья Сафроновых прожила вместе до смерти Валерия в 2028 году в автокатастрофе.
Владимир Шенкерман вынужден был эмигрировать в США в 2016 году;
он сделал неплохую карьеру в сфере информационной безопасности, женился на уроженке Мексики и прожил в Сиэттле всю оставшуюся жизнь.
