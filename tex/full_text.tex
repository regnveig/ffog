\author{Эмиль Хельгасон}
\title{Цветы для бабушки}
\date{19.08.2017}
\maketitle
\tableofcontents

\chapter{}

\section{Кошмар (первая глава)}

Вспомните свой самый худший кошмар.

Вспомнили?
Во всех подробностях?

Так вот, это ерунда.

Самый худший кошмар на свете выглядит достаточно безобидно.
Кошмар ограниченного мира.
Вариаций может быть огромное множество, и Варенику сегодня приснилась одна из них.
Звёздное небо, дом с пустыми окнами, занесённый тонким слоем снега переулок и пять могил на обочине.

Важная часть кошмара ограниченного мира "--- осознанность и кратковременная иллюзия всемогущества.
Да, я во сне, а значит, всё, что я могу придумать, может стать реальностью!
И только взлетев, распахнув крылья, ощутив биение воздуха в грудь, Вареник понял "--- не может.

Далёкие звёзды оказались простыми источниками света, подвешенными в пустоте.
Дом превратился в картонку с прорезями, а переулок "--- в единственное во всей этой карманной Вселенной место, где можно стоять и существовать.

Больше там ничего нет.

Что, если бы вы не смогли проснуться?

Варенику повезло "--- он проснулся.
Могилы и снег сменились теплом постели и живой, невероятно живой жены с огненно-рыжими волосами.
И даже мысль о сходстве карманной и реальной Вселенной его не посетила.
Ему вообще везло "--- и с мыслями, и с женой, и с самой жизнью "--- как и восьми миллиардам смертных, которым есть куда проснуться и есть куда умереть.

\razd

Ограниченный мир.
Мир стен, замков, заборов и колючей проволоки.

Вареник ненавидел колючую проволоку и заборы с острыми пиками.
Эти вещи "--- не просто архитектурные ухищрения, не просто предупреждение;
это настоящее варварство, полное презрение к человеческим страданиям.
Ни одна страна не будет свободной, пока на её территории есть колючая проволока.
В России же даже в городах, предназначенных для проживания свободных людей, каждый второй дом, каждый промышленный или торговый объект ощерились острым металлом.
Пожалуй, единственное, что не имеет границ в России "--- стремление возводить границы.

Ограниченный мир.

Законы, финансы, политика, военные зоны, частная собственность "--- удавки, наброенные на шею и не дающие сделать вдох ни на миллилитр больше, чем нужно для твоего выживания.
Живи, работай и даже не помышляй выйти за рамки.
И не думай, что эти рамки иллюзорны "--- нужно большое мужество, чтобы уйти за границы собственных финансов или привычек.

Ты не можешь построить дом там, где хочешь. 
Ты не можешь выращивать культуры и питаться от земли.
Ты не можешь перемещаться по стране, не имея целой кипы бумаг "--- документов или валюты.

Вареник желал другого мира.
Но мир только один, что бы ни говорили на этот счёт теории мультивселенных.

\section{Универ}

Универ, в котором отучилась Вероника, переживал не лучшие времена.
Раньше, говорят, все таблички на кабинетах были латунные, даже именные.
Сейчас же все они были бумажные, наскоро набранные на компьютере.
Имена и должности сменялись на них регулярно.

\section{Коробки}

"--*Ну а хуле, "--- говорил Киря прочим, "--- хочешь жить "--- умей вертеться.

Киря замечательно умел вертеться.
Он получал самую большую зарплату для его должности, его регулярно объявляли работником месяца и выдавали поощрения.
Он имел безусловный авторитет среди мужиков благодаря надёжности и честности.
Работа на складе была его коньком, а должность комплектовщика <<Ангстрема>> "--- венцом карьеры.

Вареник не хотел вертеться "--- он хотел жить.
Посему же ничем не выделялся среди других.
Ему нравился запах коробок и хруст, с которым они разрывались.
Платили ему гораздо меньше Кири, но тоже неплохо, вовремя "--- безусловное преимущество крупной компании.
Впрочем, пятое чувство подсказывало "--- пора валить.

И оно не обмануло.
Спустя пять дней после увольнения Вареник узнал, что склад влетел на крупную сумму "--- больше двадцати пяти тысяч на каждого работника.
Фура пропала вместе с водителем, грузом и кое-какими важными документами.
Долг менеджеры милостиво растянули на полгода, не оставляя и тени сомнения в великолепии этого бизнес-плана.

\section{Пирожные}

"--*Извините, вы нам не подходите.

Стандартная фраза.
Вареник слышал её уже в одиннадцатый раз.
Говорили на собеседованиях разное, но слышал он одно и то же.
Иногда проводивший собеседование эйчар даже начинал извиваться и кривиться, словно змея на сковороде;
Вареник знал "--- наступил момент, когда уже всё ясно, деньги отработаны, а желудок требует заслуженной чашки корпоративного кофе.

"--*Прощайте, "--- коротко ответил Вареник и, собрав свои бумаги, вышел из кабинета.

В этом офисе, как и в прочих других, царил декаданс.
Современные материалы, на поверку оказывающиеся отходами производства.
Запах парфюмерии, сквозь который пробивались ароматы непролеченных язв и убитых алкоголем почек.
Десятки людей с агрессивно-равнодушными взглядами, прячущими страх безработицы, ощущение собственной незначительности и два-три непогашенных кредита.

<<Может, оно и к лучшему>>, "--- решил про себя Вареник, ощущая тяжёлый взгляд охранника.
Тот крутил на пальце ключи.
Две тысячи лет назад жил человек, который с точно таким же выражением лица поигрывал кнутом.
Этого не было ни в одной хронике, ни в одном учебнике, но Вареник знал это так, словно видел своими глазами.

<<Прости, Вероника.
Хуёвый у тебя муж>>.

\section{Сигарета}

С прочими друзьями Вареник общался ровно до того момента, пока с ними было о чём покурить;
последняя сигарета истлела года два назад. 

\section{Прозвище}

Своё не очень приличное прозвище Валерка получил ещё в школе;
он достаточно рано начал ухлёстывать за девочками и имел среди них популярность, и пацаны прозвали его Вареником.
Валерка на прозвище не обиделся и и вскоре сам стал представляться так.
Потом чрезмерное увлечение противоположным полом прошло, а погоняло осталось.
Все привыкли "--- Вареник и Вареник.

\section{Несчастливая любовь}

Впрочем, были у Шенкермана и длительные отношения.
Однажды он разрывался между двумя девушками;
Варенику даже пришлось вмешаться, пока в дело не встрял алкоголь.
Всегда готовый разрулить любую ситуацию, Шенкерман был абсолютно беспомощен перед девушками и вином.

"--*Вот чем тебе нравится Светочка? "--- допытывался Вареник во время очередного сеанса психотерапии.

"--*Светочка хорошо готовит, с ней есть о чём поговорить, "--- отвечал Шенкерман после некоторого раздумья.

"--*А Олечка?

"--*Олечка картавит, "--- не задумываясь выпаливал влюблённый.

Вареник тяжко задумывался и замолкал.
Видимо, в представлении Шенкермана картавость если и не перекрывала все достоинства Светочки, то по крайней мере могла составить им серьёзную конкуренцию.

Светочка всё же выиграла соревнование с небольшим перевесом.
На целых два года Шенкерман выпал из эволюционного процесса.
Обстоятельств, по которым они разошлись, Вареник не знал;
Шенкерман начинал плеваться при любом упоминании имени <<Света>>.

\section{Грех}

"--*Знаешь, во времена Гражданской был один такой комиссар, из обедневших дворян\ldotst
Прапрадед мой со стороны отца.
В тридцатые его из-за происхождения в расход пустили по накатанной, но делов натворить он успел.
Особенно любил духовенство троллить.
Поймает очередного попа, поставит у стенки и задаёт контрольный вопрос "--- куда вошло копьё Лонгина Сотника?
И если ответит поп "--- в сердце, комиссар ему тут же пулю в печень.
А всё потому, что читать Писание бездумно, не зная анатомии "--- грех.

"--*А если поп отвечал правильно?

"--*В голову стрелял.
Дворянин всё-таки был, не пролетарий голожопый.

\section{Банкомат}

На готовку сил уже не было.
Вареник попросил у продавщицы какие-то котлеты и гарнир.
У продавщицы сил тоже не было.
Она с кислой рожей ходила туда-сюда, наполняя баночки едой.
Вареник вдруг вспомнил, как пытался подрабатывать в ресторане, надеясь на халявные харчи;
однако запах и вид \emph{не его} пищи отбивали аппетит на всю смену и до глубокой ночи. 

<<Триста пятьдесят, "--- обречённо считал Вареник, слушая писк кассового аппарата.
"--- А то и все четыреста>>.

Едва дождавшись радостного <<Покупка одобрена>>, Вареник выдернул карту, словно боялся, что жадное устройство высосет с неё все деньги.
Затем отправился к банкомату.
Банкомат принял карту, что-то промурлыкал завлекательным женским голосом, а затем задумался над надписью <<Подождите, операция выполняется>>.
Минута, две\ldotst без изменений.

У Вареника не нашлось сил даже грустно вздохнуть.

<<Блеск.
Прекрасное окончание прекрасного дня>>.

Cancel. Cancel.
Тот же результат.
Вареник на автомате вытащил телефон и набрал номер техподдержки банка.
В голове медленно проплывали мысли "--- о просроченной плате за съёмную квартиру, о задолженности по кредиту.
Ему очень хотелось сесть ничком прямо здесь и горько заплакать, но он стоял с телефоном у уха и бесстрастно слушал всё тот же мурлыкающий, завлекательный голос, говорящий о возможностях и перечисляющий цифры, которые следует нажать.

Банкомат вдруг ожил и выплюнул карту.

<<Сука>>.

Вареник подхватил карту и, едва подавив желание прописать банкомату хай-кик, отправился к соседнему.
К счастью, на этом неприятности в тот день и закончились.

\section{Взросление}

"--*Я иногда помечтать любил, "--- сказал Шенкерман.
"--- В шараге особенно. 
Знаешь, воображал себя мегакрутым воином или что-то подобное, когда типа раздеваешься до пояса и идёшь ломать руки-ноги.
И вот однажды мечтаю я себе.
Зима, снегопад, толпа народу.
И выходит против меня такой громила.
Я таких больше всего боялся "--- кого свалить у меня тупо силёнок не хватит.
Раздевается громила до пояса, мышцами играет, насмехается надо мной.
А я, вместо того, чтобы тоже раздеться, надеваю шапку и заправляю свитер в штаны.
Зима же, снегопад.

"--*И к чему это?

"--*Да так, "--- ухмыльнулся Шенкерман.
"--- Я просто в тот момент понял, что повзрослел.

\section{Жена}

Существует забавная теория о громоотводах.
У человечества есть громоотводы ненависти, к коим, безусловно, любил относить себя Шенкерман "--- всякий раз, когда его называли <<жидопидарасом>>, что случалось явно чаще честных президенстких выборов в России.
Вареник скептически относился к словам приятеля, но в существование громоотвода грусти поверил бы без лишних слов.
Одним из них была его жена, Вероника, и судя по объёму грусти, который она успешно утилизировала, Земля не утопала в печали благодаря не более чем десяти столпам.

Вероника работала в <<Векторе>>, новосибирском центре вирусологии и биотехнологии.
Вареник мало знал о том, чем именно занимается жена;
она предпочитала не распространяться.
Гораздо больше Вероника рассказывала о коллегах "--- у кого какие проблемы, кто как живёт.
По-видимому, тихая и чуткая девушка служила на предприятии нештатным психологом "--- ей рассказывали всё, от подробностей рождения до самых страшных личных тайн.
Иногда после очередного сеанса психотерапии Вероника приходила домой хмурая, и Вареник знал "--- столп Земли нужно поддержать.

"--*Опять передозировка? "--- шутил он, обнимая жену перед сном.

Вероника поглубже зарывалась носом мужу в подмышку и молча засыпала.
Наутро всё как рукой снимало;
а вот Вареник мучился кошмарами целую неделю и мысленно благодарил небеса, что эту странную роль "--- роль громоотвода "--- Судьба назначила не ему.

\section{Несуществующий город}

"--*А прикол, ля, в том, что города этого, ля, ни на картах нету, ни в справочниках! "--- Шенкерман развёл руками.

"--*Ты уверен, Володь?

"--*Зайди в гугльмапс, ля!
Какой в опу Зеленообск?
На этом месте лес и голая трасса, ни поворота, ни города!

"--*Напиши в техподдержку.
Не может быть.
Вот карта автодорог семьдесят пятого года, город на месте.

"--*Сейчас два ноль двенадцатый, ля.
И гуглю я доверяю больше, ля, чем твоим совковым картам.
Те малость устарели, ля.

"--*Но ты же сам там был, Володька!

"--*Мало ли где я был, "--- буркнул Шенкерман.
"--- Не мог целый город, ля, вот так взять и испариться. 
Значит, либо мы были в другом месте, либо\ldotst на этом мысль останавливается. 
Ну сам подумай, ля, "--- мы у кого-то спросили название города?
Нет.
Приехали утром, зашли к бабушке, подарили цветы.
Всё, ля.

"--*А адрес?
Улица Фрунзе, дом\ldotst

"--*Вареник, не тупи.
Улица Фрунзе, ля, бывает везде.

"--*Да, точно, сорян.
А почему все были в курсе, что мы приедем?

"--*А откуда я знаю, ля?
Акция всероссийская!

Вареник был вынужден признать, что друг прав.
Мистики ноль. 
Они просто ранним утречком приехали неизвестно куда и подарили цветы неизвестно кому.

\section{Карьера}

Своей карьерой Вареник был обязан порезанному пальцу.
В тот знаменательный день он окончательно поругался с матерью, и та отказалась высылать ему деньги.
Вареник не особо расстроился.
Он расстроился бы, если бы у него были на это душевные силы "--- поиск работы шёл не ахти как, деньги заканчивались, да и ссора с матерью не была первой в его жизни.
<<Заебала>>, "--- вздохнул Вареник и пошёл мыться.

Палец резануло неожиданно.
Вареник всегда гордился своей реакцией;
падающая зубная щётка не успела коснуться пола.
Однако стеклянная полочка имела своё видение ситуации.
Палец залил кровью все поверхности, словно обезглавленный неумелым палачом преступник;
на срезе проступили сухожилия "--- к счастью, целые.

Вареник собирался после душа поискать работу, но палец окончательно его доконал. 
Он залил рану какой-то дрянью, попытался свести воедино края кожи и перемотал злосчастный палец бинтом;
затем, чтобы успокоиться, без особых надежд занялся старым проектом, который безуспешно пытался раскрутить последние два года.

Проект взлетел в тот же день.
Финансовые проблемы на время ушли в прошлое.

\section{Призрак}

Есть нечто прикольное в том, чтобы любить призрака.
Призрак никогда тебя не разочарует.
Призрак не исчерпает себя как личность, ведь он "--- твоё продолжение с лицом другого человека, и глубина призрака равна твоей глубине.
К призраку не нужно стремиться.
А в самые тяжёлые моменты можно с безобидным эгоизмом мечтать, что есть кто-то, кто без тебя никогда не будет счастлив, кто всегда тебя знал и ждал.
И эта фраза "--- глупая, на самом деле "--- всегда будет на самом краю той не требующей реализации мечты: <<Где ты, сука, всё это время был?>>

У Вареника такой призрак имелся.
Даже сейчас, после пяти лет в браке по большой любви, он иногда доставал телефон и долго смотрел на фото десятилетней девочки, игравшей на пианино.
Вероника думала, что это его троюродная племянница.
На самом деле племяннице было уже под тридцатник, и жила она в далёком Берлине с мужем и детьми.
Вареника не мучила совесть.
Он считал, что у каждого человека есть то сокровенное, о котором не стыдно врать "--- даже самым близким людям.

Никто, кроме тебя, не имеет прав на твоего призрака.

\section{Архивы}

Вареник лихорадочно листал страницы пожелтевшего дневника.

<<17 июля 1954 года.
Пришёл день, когда тяйшхаэры обещали посвятить меня в тайну долголетия.
Сайры-хаэр привёл меня домой.
Он попросил, чтобы я протянула руку, сделал надрез и выпил немного моей крови.
Я не понимаю, что произошло "--- из-за этого пореза племя стало относиться ко мне едва ли не со священным трепетом\ldotst>>

<<26 августа 1954 года.
Со мной что-то происходит.
Я начала худеть, и эта немного болезненная худоба напоминает мне девическую.
Моё лицо побледнело, пропали морщины, глаза стали просто огромными.
Мне ужасно хочется крови (зачёркнуто) кро (закрашено полностью)
Я подозвала Бердышку и сделала ему небольшой порез, но это лишь распалило жажду.
Боюсь, что\ldotst>>

<<29 августа 1954 года.
Всё отлично.
Сайры-хаэр "--- замечательный человек. 
Нужна человеческая кровь.
Да, это звучит ужасно, но её нужна какая-то капля.
Я объяснила ситуацию девочкам, и Тоня дала мне каплю своей крови.
Их поразила произошедшая во мне перемена.
Они тоже хотят вечной молодости.
Сайры-хаэр рассказал, что вечную молодость мы сможем поддерживать, только если рядом будет один ,,простой человек``.
Иначе мы все умрём от жажды>>.

<<6 сентября 1954 года.
Я в ярости.
Тоня и Света, не посоветовавшись со мной, получили ,,посвящение``.
Они довольны "--- молодость возвращается к ним.
У Тони даже пропал рубец от удаления аппендикса.
Но что делать?!
Как мы вернёмся обратно?!
Как мы объясним свою ежемесячную жажду крови?!
Свету это не волнует.
Она сказала, что её муж всё вытерпел, и каплю крови для неё не пожалеет.
Чёртовы дуры!
Они не понимают, чем это обернётся!
Из-за этой странной заразы мечты Владимира Ильича пойдут прахом.
Люди разделятся на два класса, как тяйшхаэры "--- не знающие болезней сверхлюди и все прочие, служащие лишь источником здоровой крови.
Мы станем угнетателями, не лучше капиталистов.
Но девчонкам плевать, они думают лишь о себе.
Если бы я могла\ldotst если бы я только могла вернуться и отказаться от этого ужасного дара!>>

\section{Жадность}

Нет, не прав был Иешуа Га-Ноцри.
Страшнейший из пороков "--- отнюдь не трусость. 
Жадность Иуды до денег, жадность Синедриона до власти отправили Иешуа на крест, а отнюдь не старый, сломленный службой, одиночеством и болезнью прокуратор.
Жадность "--- вот самый страшный порок, и едва ли кто-то усомнится в этом, глядя на сегодняшнюю Россию "--- безмерно богатую и бесконечно нищую, словно профессиональный попрошайка.

\section{Начало войны}

"--*Сюда уже едут, "--- сказала Зинаида Павловна.
"--- Они не должны никого найти.
Никого живого и тёплого.

"--*И что ты предлагаешь, Зинка? "--- осведомилась одна из женщин.

"--*Я предлагаю вам добровольно вернуть долг за долгую здоровую жизнь.

Поднялся гвалт.

"--*Да иди ты в пизду, полоумная!

"--*Она права!

"--*Да где права-то?
Бежать надо!

"--*Если кто-то хочет бежать "--- собирайте вещи, "--- сказала Зинаида Павловна.
"--- Я вам больше не командир.

Толпа разделилась;
больше двух третей ушли.

"--*Отлично.
Теперь окончательно выяснили, кто из нас порядочный человек, а кто нет.

"--*Так что делать-то, Зин? "--- спросила Глафира.

"--*Да известно что.
Не хотят долги возвращать "--- будем взыскивать.
Глаш, ты у нас фронтовая, можешь девкам объяснить, что да как?

Глафира обратилась к остальным:

"--*Нина, Маша "--- доставайте своё из загашника, смазать не забудьте.
Остальные "--- найдите ломы да ножи понадёжнее.
Начнём с этой общаги, с нижнего этажа.
Вы двое "--- на пожарную лестницу у входа, вы четверо "--- на заднюю.
Пока все на позициях не будут "--- молчим и глазами не сверкаем.
Всем ясно? 

\section{Война}

Вареник обернулся и увидел Глафиру.
От ворчливой старушки не осталось ничего "--- седые волосы лежали на плечах, словно у древней норны, голубые глаза горели неистовым огнём фанатика-берсерка.
Её рубаха и лицо были в крови, словно она разрывала тела врагов голыми руками.
Войдя в комнату, Глафира чинно повернулась к образам и трижды осенила себя крестным знаменем.

"--*Готово, Зин, "--- хмуро сказала она, вытирая нож.
"--- Тебя\ldotsq

"--*Не нужно, "--- остановила подругу Зинаида Петровна.
"--- Прошу тебя, возьми ещё грех на душу "--- помоги товаркам и себе.

"--*Да какой грех, Господи, "--- сказала Глафира.
"--- Коли не стану душегубкой, так Иисус Христос, Сын Божий, с меня вчетверо спросит "--- за то, что позволила жить лукавому.
Идём, девочки.

"--*Не, Глаха, я сама, "--- проскрипела Нина.
"--- Рука у тебя верная, что ни говори, но за себя сама отвечу, не обессудь.

\section{Вероника}

"--*Я уволилась из <<Вектора>>, "--- пробормотала Вероника.
"--- Сказала, что устала и устроюсь в институте, поближе к дому и поспокойнее.

"--*Они не знают?

"--*Нет.

\section{6}

Если за вечную жизнь приходится платить тем, что сам ты сотворить не в состоянии "--- это вечное рабство, а не вечная жизнь.

\section{Имя}

"--*Вареник, а как тебя всё-таки звать?

"--*Володь, ты издеваешься? "--- опешил собеседник.

"--*Я серьёзно, "--- Шенкерман выглядел слегка растерянным.
"--- Сколько тебя помню, ты всегда был Вареником.
А кто ты по паспорту-то?
Виталий?

"--*Валерий.
Валерий Николаевич.

"--*Ты потратил пять секунд на то, чтобы вспомнить, "--- заметил Шенкерман. 
"--- Как прозвище-то въедается.

"--*Неправда.
Ты меня просто ошарашил вопросом.

"--*А давно тебя по имени-отчеству величали-то?

"--*А вот чёрт знает.
В политехе вроде.
У нас препод был смешной, всех звал <<мастерами>>.
Я был у него <<мастер Сафронов>> или <<Валерий Николаевич>>.

"--*Он тебе шанс давал, Вареник, "--- ухмыльнулся Шенкерман.

"--*Какой шанс?

"--*Недостаточно родиться Валерием Николаевичем.
Им надо стать.

"--*Ещё немного "--- и я стану антисемитом, "--- предупредил друга Вареник.

"--*Да хоть в скинхеды иди, "--- отмахнулся Шенкерман.
"--- Ты данные по вирусу принёс или мне гороскоп составлять для поиска?

\section{Смысл без любви}

Шенкерман лежал и думал, что же не так с этой больницей?
Ответ пришёл постепенно "--- безвкусные стены были вымерены до сантиметра, по нормам СанПиНов;
над каждой розеткой висела столь же безвкусная табличка, сделанная исключительно из-за требований вышестоящих органов.
А кафель, который покрывал стену позади раковины, совершенно не сочетался по расцветке с основной стеной.
В каждой детали был смысл, но не было ни грамма любви.

<<Как и во всей моей грёбаной жизни, "--- грустно подытожил Шенкерман.
"--- К чёрту.
Разберусь с переломом "--- и на Родину\ldotst
Хватит с меня этой российской действительности>>.

\section{Конец}

"--*Извините, Зинаида Павловна, что так вышло.

"--*Не извиняйся, милый.
Надо было, конечно, сразу обо всём рассказать\ldotst
У нас ещё были идеалы "--- служить Родине, пролетариату\ldotst да вот знала я, что с нами сделают, коль проболтаемся.

"--*Вы думаете, вас бы убили?

"--*Хуже, милый.
У моей матери, которая в Поволжье жила, как-то тётка была.
Поехала в город на работу, а в родной деревне авария.
Сошёл с рельс поезд с какими-то химикатами, деревня вспыхнула, как спичка\ldotst
Неизвестно, остался ли кто в живых, но сверху велели согнать машины да запахали пепелище. 
Приехала тётка "--- а деревни нет.
Поехала узнавать в город "--- а её никогда и не было.
В лечебницу забрали и мытарили там пять лет\ldotst

Вареник почувствовал, что плачет.

"--*И войну братоубийственную запахали.
Муж мой сам пошёл.
Мог не идти, да пошёл.
Написал как-то, что у немцев много солдат с дырками в спине.
От <<Вальтера>>.
Не трусы были, честные вояки.
А потом сам вернулся с пулей в спине "--- уже от ТТ.
Комиссар мне потом так и сказал: <<Скажи спасибо, красавица, что мы его тебе Героем Союза привезли.
Могли и по-другому>>.
А потом его победой "--- его Победой! "--- запахали сталинские грехи. 
И Чернобыль бы с Припятью так же запахали, коль можно было бы.
Ты уж мне поверь. 
Сейчас всё ж другие времена "--- интернет есть.
Сейчас не запашут\ldotst

Зинаида Павловна закашлялась.
Вареник поднёс ей чашку полуостывшего чая.

"--*Идём, милый, я тебя причешу, "--- Зинаида Павловна улыбнулась мужчине вымученной улыбкой.
"--- Не бойся, просто причешу.
Всё одно сегодня умру\ldotst
Глашка говорит, что Богу душу отдают люди, да мы-то не такие, мы люди советские, атеисты.
Знаешь, какая школа-то у нас была?
Ууу\ldotst
Да иди, не бойся.

Вареник, поколебавшись, подошёл и сел рядом с кроватью.
Зинаина Павловна взяла гребень "--- не железный, а обычный, деревянный.
Вареник закрыл глаза.
Он ощущал себя котом.
Как же всё-таки здорово, когда тебя чешут.

"--*Ну вот, "--- старушка отложила гребень в сторону и потрепала мужчину по щеке.
"--- Теперь иди.
Веронике привет от меня передавай.
Так и скажи ей "--- она последняя.
Будет благоразумной "--- последней и останется.

"--*Она уже хотела\ldotst

"--*Рано хотела.
Ну свели девки счёты с жизнью, ну и что с того?
Они-то пожили.
Вероника тоже пускай поживёт.
Всё одно, как ни крути "--- долгая молодость, здоровая жизнь, счастливая старость.
Всё одно дар, хоть и проклятый.
У вас ещё молодость-то прекраснее нашей "--- столько соблазнов, столько удовольствий, мир как на ладони.
Грех такую терять.

"--*А дети?

"--*Детей усыновите.
Много их нынче, сироток.
Кто, если не вы?

Вареник закинул сумку на плечо и проверил, на месте ли ключи от машины.

"--*Так значит, вы скоро умрёте? "--- обернулся Вареник к Зинаиде Павловне. 
"--- А откуда вы это знаете?

Но она уже не дышала.
