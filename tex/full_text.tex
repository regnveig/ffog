\author{Эмиль Хельгасон}
\title{Цветы для бабушки}
\date{19.08.2017}
\maketitle

\begin{abstract}
Две тысячи двенадцатый год, Россия.
У Валерия Сафронова, обычного новосибирца, жена заболела странной болезнью.
Он готов на всё, чтобы вылечить её.
Однако вскоре оказывается "--- болезнь гораздо опаснее, чем он мог представить\ldotst
\end{abstract}

\tableofcontents

\section{Кошмар (первая глава)}

Вспомните свой самый худший кошмар.

Вспомнили?
Во всех подробностях?

Так вот, это ерунда.

Самый худший кошмар на свете выглядит достаточно безобидно.
Кошмар ограниченного мира.
Вариаций может быть огромное множество, и Варенику сегодня приснилась одна из них.
Звёздное небо, дом с пустыми окнами, занесённый тонким слоем снега переулок и пять могил на обочине.

Важная часть кошмара ограниченного мира "--- осознанность и кратковременная иллюзия всемогущества.
Да, я во сне, а значит, всё, что я могу придумать, может стать реальностью!
И только взлетев, распахнув крылья, ощутив биение воздуха в грудь, Вареник понял "--- не может.

Далёкие звёзды оказались простыми источниками света, подвешенными в пустоте.
Дом превратился в картонку с прорезями, а переулок "--- в единственное во всей этой карманной Вселенной место, где можно стоять и существовать.

Больше там ничего нет.

Что, если бы вы не смогли проснуться?

Варенику повезло "--- он проснулся.
Могилы и снег сменились теплом постели и живой, невероятно живой жены с огненно-рыжими волосами.
И даже мысль о сходстве карманной и реальной Вселенной его не посетила.
Ему вообще везло "--- и с мыслями, и с женой, и с самой жизнью "--- как и восьми миллиардам смертных, которым есть куда проснуться и есть куда умереть.

\razd

Ограниченный мир.
Мир стен, замков, заборов и колючей проволоки.

Вареник ненавидел колючую проволоку и заборы с острыми пиками.
Эти вещи "--- не просто архитектурные ухищрения, не просто предупреждение;
это настоящее варварство, полное презрение к человеческим страданиям.
Ни одна страна не будет свободной, пока на её территории есть колючая проволока.
В России же даже в городах, предназначенных для проживания свободных людей, каждый второй дом, каждый промышленный или торговый объект ощерились острым металлом.
Пожалуй, единственное, что не имеет границ в России "--- стремление возводить границы.

Ограниченный мир.

Законы, финансы, политика, военные зоны, частная собственность "--- удавки, наброенные на шею и не дающие сделать вдох ни на миллилитр больше, чем нужно для твоего выживания.
Живи, работай и даже не помышляй выйти за рамки.
И не думай, что эти рамки иллюзорны "--- нужно большое мужество, чтобы уйти за границы собственных финансов или привычек.

Ты не можешь построить дом там, где хочешь. 
Ты не можешь выращивать культуры и питаться от земли.
Ты не можешь перемещаться по стране, не имея целой кипы бумаг "--- документов или валюты.

Вареник желал другого мира.
Но мир только один, что бы ни говорили на этот счёт теории мультивселенных.

\section{Универ}

Универ, в котором отучилась Вероника, переживал не лучшие времена.
Раньше, говорят, все таблички на кабинетах были латунные, даже именные.
Сейчас же все они были бумажные, наскоро набранные на компьютере.
Имена и должности сменялись на них регулярно.

\section{Коробки}

"--*Ну а хуле, "--- говорил Киря прочим, "--- хочешь жить "--- умей вертеться.

Киря замечательно умел вертеться.
Он получал самую большую зарплату для его должности, его регулярно объявляли работником месяца и выдавали поощрения.
Он имел безусловный авторитет среди мужиков благодаря надёжности и честности.
Работа на складе была его коньком, а должность комплектовщика <<Ангстрема>> "--- венцом карьеры.

Вареник не хотел вертеться "--- он хотел жить.
Посему же ничем не выделялся среди других.
Ему нравился запах коробок и хруст, с которым они разрывались.
Платили ему гораздо меньше Кири, но тоже неплохо, вовремя "--- безусловное преимущество крупной компании.
Впрочем, пятое чувство подсказывало "--- пора валить.

И оно не обмануло.
Спустя пять дней после увольнения Вареник узнал, что склад влетел на крупную сумму "--- больше двадцати пяти тысяч на каждого работника.
Фура пропала вместе с водителем, грузом и кое-какими важными документами.
Долг менеджеры милостиво растянули на полгода, не оставляя и тени сомнения в великолепии этого бизнес-плана.

\section{Пирожные}

"--*Извините, вы нам не подходите.

Стандартная фраза.
Вареник слышал её уже в одиннадцатый раз.
Говорили на собеседованиях разное, но слышал он одно и то же.
Иногда проводивший собеседование эйчар даже начинал извиваться и кривиться, словно змея на сковороде;
Вареник знал "--- наступил момент, когда уже всё ясно, деньги отработаны, а желудок требует заслуженной чашки корпоративного кофе.

"--*Прощайте, "--- коротко ответил Вареник и, собрав свои бумаги, вышел из кабинета.

В этом офисе, как и в прочих других, царил декаданс.
Современные материалы, на поверку оказывающиеся отходами производства.
Запах парфюмерии, сквозь который пробивались ароматы непролеченных язв и убитых алкоголем почек.
Десятки людей с агрессивно-равнодушными взглядами, прячущими страх безработицы, ощущение собственной незначительности и два-три непогашенных кредита.

<<Может, оно и к лучшему>>, "--- решил про себя Вареник, ощущая тяжёлый взгляд охранника.
Тот крутил на пальце ключи.
Две тысячи лет назад жил человек, который с точно таким же выражением лица поигрывал кнутом.
Этого не было ни в одной хронике, ни в одном учебнике, но Вареник знал это так, словно видел своими глазами.

<<Прости, Вероника.
Хуёвый у тебя муж>>.

\section{Сигарета}

С прочими друзьями Вареник общался ровно до того момента, пока с ними было о чём покурить;
последняя сигарета истлела года два назад. 

\section{Прозвище}

Своё не очень приличное прозвище Валерка получил ещё в школе;
он достаточно рано начал ухлёстывать за девочками и имел среди них популярность, и пацаны прозвали его Вареником.
Валерка на прозвище не обиделся и и вскоре сам стал представляться так.
Потом чрезмерное увлечение противоположным полом прошло, а погоняло осталось.
Все привыкли "--- Вареник и Вареник.

\section{Несчастливая любовь}

Впрочем, были у Шенкермана и длительные отношения.
Однажды он разрывался между двумя девушками;
Варенику даже пришлось вмешаться, пока в дело не встрял алкоголь.
Всегда готовый разрулить любую ситуацию, Шенкерман был абсолютно беспомощен перед девушками и вином.

"--*Вот чем тебе нравится Светочка? "--- допытывался Вареник во время очередного сеанса психотерапии.

"--*Светочка хорошо готовит, с ней есть о чём поговорить, "--- отвечал Шенкерман после некоторого раздумья.

"--*А Олечка?

"--*Олечка картавит, "--- не задумываясь выпаливал влюблённый.

Вареник тяжко задумывался и замолкал.
Видимо, в представлении Шенкермана картавость если и не перекрывала все достоинства Светочки, то по крайней мере могла составить им серьёзную конкуренцию.

Светочка всё же выиграла соревнование с небольшим перевесом.
На целых два года Шенкерман выпал из эволюционного процесса.
Обстоятельств, по которым они разошлись, Вареник не знал;
Шенкерман начинал плеваться при любом упоминании имени <<Света>>.

\section{Дорога}

Вероника уже давно спала, откинув голову на сиденье и открыв рот.

"--*Слушай, вот зачем тебе это понадобилось, а? "--- сонным голосом спросил Шенкерман.
"--- Я не выспался\ldotst

"--*А я тебя предупреждал, что завтра едем.
Цветы не помни.

"--*Ладно, ладно!

Уже на трассе Вареник вспомнил, что забыл посмотреть адрес.

"--*Так, третий поворот должен быть, там дальше по картам посмотрим.

"--*Я телефон не взял, "--- пожаловался Шенкерман.

"--*Молодец, подготовился.

Вареник вытащил свой.

<<Критическое обновление.
Пожалуйста, подождите, установка может занять\ldotst>>

"--*Сука, да вы издеваетесь.

"--*Что ещё?

"--*Обновление, мать его.
Вообще нужно спрашивать мнение пользователя, удобно ли ему\ldotst

"--*Да кому всралось твоё мнение?
У Гугла бизнес-план горит.

Вареник громко выругался и забросил телефон в бардачок.

"--*Нику будить?

"--*Добудишься её сейчас.
Ника!
Солнце ты моё!
Телефон твой нужен!

Ответом был короткий грудной всхрап.

"--*Мда.

"--*Давай я у неё в кармане пошарю, "--- предложил Шенкерман.

"--*Так, блядь, никакой акробатики у меня в машине!
Тем более на трассе!

"--*Ну а хуле делать тогда?

"--*Хуле делать?
По старинке будем действовать "--- останавливаться и спрашивать людей.

"--*Кого мы встретим в восемь часов утра, в воскресенье?

"--*Таких же дебилов, как и мы, Володька, только с работающими, блядь, телефонами.

"--*Я надеялся, что ты скажешь <<Повернём назад и поедем досыпать>>.

"--*Не дождёшься.
Букеты уже купили, блин.
Поздно уже, сука.

"--*Какая гадость эта ваша благотворительность.

"--*Заткнись, а?

Вареник дёрнул коробку передач и раздражённо вдавил педаль газа.

\section{Грех}

"--*Знаешь, во времена Гражданской был один такой комиссар, из обедневших дворян\ldotst
Прапрадед мой со стороны отца.
В тридцатые его из-за происхождения в расход пустили по накатанной, но делов натворить он успел.
Особенно любил духовенство троллить.
Поймает очередного попа, поставит у стенки и задаёт контрольный вопрос "--- куда вошло копьё Лонгина Сотника?
И если ответит поп "--- в сердце, комиссар ему тут же пулю в печень.
А всё потому, что читать Писание бездумно, не зная анатомии "--- грех.

"--*А если поп отвечал правильно?

"--*В голову стрелял.
Дворянин всё-таки был, не пролетарий голожопый.

\section{Банкомат}

На готовку сил уже не было.
Вареник попросил у продавщицы какие-то котлеты и гарнир.
У продавщицы сил тоже не было.
Она с кислой рожей ходила туда-сюда, наполняя баночки едой.
Вареник вдруг вспомнил, как пытался подрабатывать в ресторане, надеясь на халявные харчи;
однако запах и вид \emph{не его} пищи отбивали аппетит на всю смену и до глубокой ночи. 

<<Триста пятьдесят, "--- обречённо считал Вареник, слушая писк кассового аппарата.
"--- А то и все четыреста>>.

Едва дождавшись радостного <<Покупка одобрена>>, Вареник выдернул карту, словно боялся, что жадное устройство высосет с неё все деньги.
Затем отправился к банкомату.
Банкомат принял карту, что-то промурлыкал завлекательным женским голосом, а затем задумался над надписью <<Подождите, операция выполняется>>.
Минута, две\ldotst без изменений.

У Вареника не нашлось сил даже грустно вздохнуть.

<<Блеск.
Прекрасное окончание прекрасного дня>>.

Cancel. Cancel.
Тот же результат.
Вареник на автомате вытащил телефон и набрал номер техподдержки банка.
В голове медленно проплывали мысли "--- о просроченной плате за съёмную квартиру, о задолженности по кредиту.
Ему очень хотелось сесть ничком прямо здесь и горько заплакать, но он стоял с телефоном у уха и бесстрастно слушал всё тот же мурлыкающий, завлекательный голос, говорящий о возможностях и перечисляющий цифры, которые следует нажать.

Банкомат вдруг ожил и выплюнул карту.

<<Сука>>.

Вареник подхватил карту и, едва подавив желание прописать банкомату хай-кик, отправился к соседнему.
К счастью, на этом неприятности в тот день и закончились.

\section{Никита}

Взять какого-нибудь человека.
У него есть план, как и у миллионов других "--- школа, университет, работа, дети.
Он совершенно обычный "--- вежливый, в костюмчике, занимается тем же, чем все, отдыхает так же, как все.
Дела расписаны чуть ли не по дням и часам.

Но в один момент он делает что-то не так.
Например, после универа в родной стране он бросает девушку и едет в холодную Сибирь получать второе высшее.
И в этот момент, когда он принял решение, обратной дороги нет "--- его судьба обречена быть не такой, как у других.
Потому что к нему начинают относиться, как к особенному человеку, он начинает притягивать особенных людей.

Иногда кажется, что его жизнь катится ко всем чертям, но это только прибавляет ей индивидуальности.
Этот человек будет жить по-другому, он будет счастлив, он будет искать свой путь окольными тропами.
И однажды, возможно, он протопчет дорогу специально для себя.

А люди, которые жили по шаблону, будут несчастны по тому же шаблону.
Потому что счастья по шаблону не бывает.

\section{Кошка и мышь}

"--*Загадка: как найти чёрную кошку и чёрную мышь в тёмной комнате?

"--*И как?

"--*По хрусту.

"--*Не понял.

"--*Скоро поймёшь.

\section{Звезда по имени}

"--*Ну не знаю.
Я всё ещё верю, что где-то там надо мной светит моя счастливая звезда.
Осталось только её найти.

"--*Эта звезда светит всем, и искать её не надо.
Она называется Солнце.

\section{Взросление}

"--*Я иногда помечтать любил, "--- сказал Шенкерман.
"--- В шараге особенно. 
Знаешь, воображал себя мегакрутым воином или что-то подобное, когда типа раздеваешься до пояса и идёшь ломать руки-ноги.
И вот однажды мечтаю я себе.
Зима, снегопад, толпа народу.
И выходит против меня такой громила.
Я таких больше всего боялся "--- кого свалить у меня тупо силёнок не хватит.
Раздевается громила до пояса, мышцами играет, насмехается надо мной.
А я, вместо того, чтобы тоже раздеться, надеваю шапку и заправляю свитер в штаны.
Зима же, снегопад.

"--*И к чему это?

"--*Да так, "--- ухмыльнулся Шенкерман.
"--- Я просто в тот момент понял, что повзрослел.

\section{Жена}

Существует забавная теория о громоотводах.
У человечества есть громоотводы ненависти, к коим, безусловно, любил относить себя Шенкерман "--- всякий раз, когда его называли <<жидопидарасом>>, что случалось явно чаще честных президенстких выборов в России.
Вареник скептически относился к словам приятеля, но в существование громоотвода грусти поверил бы без лишних слов.
Одним из них была его жена, Вероника, и судя по объёму грусти, который она успешно утилизировала, Земля не утопала в печали благодаря не более чем десяти столпам.

Вероника работала в <<Векторе>>, новосибирском центре вирусологии и биотехнологии.
Вареник мало знал о том, чем именно занимается жена;
она предпочитала не распространяться.
Гораздо больше Вероника рассказывала о коллегах "--- у кого какие проблемы, кто как живёт.
По-видимому, тихая и чуткая девушка служила на предприятии нештатным психологом "--- ей рассказывали всё, от подробностей рождения до самых страшных личных тайн.
Иногда после очередного сеанса психотерапии Вероника приходила домой хмурая, и Вареник знал "--- столп Земли нужно поддержать.

"--*Опять передозировка? "--- шутил он, обнимая жену перед сном.

Вероника поглубже зарывалась носом мужу в подмышку и молча засыпала.
Наутро всё как рукой снимало;
а вот Вареник мучился кошмарами целую неделю и мысленно благодарил небеса, что эту странную роль "--- роль громоотвода "--- Судьба назначила не ему.

\section{Несуществующий город}

"--*А прикол, ля, в том, что города этого, ля, ни на картах нету, ни в справочниках! "--- Шенкерман развёл руками.

"--*Ты уверен, Володь?

"--*Зайди в гугльмапс, ля!
Какой в опу Зеленообск?
На этом месте лес и голая трасса, ни поворота, ни города!

"--*Напиши в техподдержку.
Не может быть.
Вот карта автодорог семьдесят пятого года, город на месте.

"--*Сейчас два ноль двенадцатый, ля.
И гуглю я доверяю больше, ля, чем твоим совковым картам.
Те малость устарели, ля.

"--*Но ты же сам там был, Володька!

"--*Мало ли где я был, "--- буркнул Шенкерман.
"--- Не мог целый город, ля, вот так взять и испариться. 
Значит, либо мы были в другом месте, либо\ldotst на этом мысль останавливается. 
Ну сам подумай, ля, "--- мы у кого-то спросили название города?
Нет.
Приехали утром, зашли к бабушке, подарили цветы.
Всё, ля.

"--*А адрес?
Улица Фрунзе, дом\ldotst

"--*Вареник, не тупи.
Улица Фрунзе, ля, бывает везде.

"--*Да, точно, сорян.
А почему все были в курсе, что мы приедем?

"--*А откуда я знаю, ля?
Акция всероссийская!

Вареник был вынужден признать, что друг прав.
Мистики ноль. 
Они просто ранним утречком приехали неизвестно куда и подарили цветы неизвестно кому.

Однако, вернувшись домой, Вареник раз за разом в мыслях возвращался к душераздирающему пейзажу, который встретил их в Зеленообске.
Дома-<<пустышки>>, поросшие бурьяном бетонные плиты на дорогах, отсутствие неоновых вывесок и вообще какого-либо освещения.
Несмотря на почти полное отсутствие прохожих, нужную улицу друзья нашли без труда.
Как подозревал Вареник, она была единственной в городе.

Подобное запустение было чем-то из ряда вон выходящим, даже для России.
Этот город мог существовать на Курилах, в якутской тайге или ином Богом забытом месте.
Но не в центре Новосибирской области.

Впрочем, возложенная фондом миссия была успешно выполнена, и новоявленный Сайлент-Хилл забылся за чередой обычных забот.

\section{Карьера}

Своей карьерой Вареник был обязан порезанному пальцу.
В тот знаменательный день он окончательно поругался с матерью, и та отказалась высылать ему деньги.
Вареник не особо расстроился.
Он расстроился бы, если бы у него были на это душевные силы "--- поиск работы шёл не ахти как, деньги заканчивались, да и ссора с матерью не была первой в его жизни.
<<Заебала>>, "--- вздохнул Вареник и пошёл мыться.

Палец резануло неожиданно.
Вареник всегда гордился своей реакцией;
падающая зубная щётка не успела коснуться пола.
Однако стеклянная полочка имела своё видение ситуации.
Палец залил кровью все поверхности, словно обезглавленный неумелым палачом преступник;
на срезе проступили сухожилия "--- к счастью, целые.

Вареник собирался после душа поискать работу, но палец окончательно его доконал. 
Он залил рану какой-то дрянью, попытался свести воедино края кожи и перемотал злосчастный палец бинтом;
затем, чтобы успокоиться, без особых надежд занялся старым проектом, который безуспешно пытался раскрутить последние два года.

Проект взлетел в тот же день.
Финансовые проблемы на время ушли в прошлое.

\section{Призрак}

Есть нечто прикольное в том, чтобы любить призрака.
Призрак никогда тебя не разочарует.
Призрак не исчерпает себя как личность, ведь он "--- твоё продолжение с лицом другого человека, и глубина призрака равна твоей глубине.
К призраку не нужно стремиться.
А в самые тяжёлые моменты можно с безобидным эгоизмом мечтать, что есть кто-то, кто без тебя никогда не будет счастлив, кто всегда тебя знал и ждал.
И эта фраза "--- глупая, на самом деле "--- всегда будет на самом краю той не требующей реализации мечты: <<Где ты, сука, всё это время был?>>

У Вареника такой призрак имелся.
Даже сейчас, после пяти лет в браке по большой любви, он иногда доставал телефон и долго смотрел на фото десятилетней девочки, игравшей на пианино.
Вероника думала, что это его троюродная племянница.
На самом деле племяннице было уже под тридцатник, и жила она в далёком Берлине с мужем и детьми.
Вареника не мучила совесть.
Он считал, что у каждого человека есть то сокровенное, о котором не стыдно врать "--- даже самым близким людям.

Никто, кроме тебя, не имеет прав на твоего призрака.

\section{Матримониальные планы}

"--*К нам в гости придут тетя Лиля и дядя Саша Фельдман с дочкой Лиорой.
Милая девочка, черноглазая, кудри по пояс, медицинский заканчивает в будущем году!
Я вас познакомлю, ма ихпат лахэм?

"--*Бабушка, я к вам обязательно приеду, но пожалуйста, давайте обойдёмся без матримониальных схем!
У меня есть девушка, не еврейка, но я её люблю.
Кристина, я про неё рассказывал.

"--*Пока я её не увидела, у тебя нет девушки! "--- крупные зубы бабушки сияли даже в телефонном разговоре.
"--- Шучу, не волнуйся.
Понимаю, сейчас времена другие, не то что раньше\ldotst
В четверг ха-Ацерет, приезжай в пятницу, дома будет только Яша, я что-нибудь тебе приготовлю.
Я знаю, что ты атеист, просто хочу тебя повидать.

Насчёт девушки Шенкерман соврал.
Бабушка терпимо относилась к атеизму, но отсутствие у внука личной жизни приводило её в исступление.
Если бы не придуманная девушка, знакомство с Лиорой Фельдман произошло бы в добровольно-принудительном порядке.
Прецеденты были.

\section{Архивы}

Вареник лихорадочно листал страницы пожелтевшего дневника.

<<17 июля 1929 года.
Пришёл день, когда тяйшхаэры обещали посвятить меня в тайну долголетия.
Сайры-хаэр привёл меня домой.
Он попросил, чтобы я протянула руку, сделал надрез и выпил немного моей крови.
Я не понимаю, что произошло "--- из-за этого пореза племя стало относиться ко мне едва ли не со священным трепетом\ldotst>>

<<26 августа 1929 года.
Со мной что-то происходит.
Я начала худеть, и эта немного болезненная худоба напоминает мне девическую.
Моё лицо побледнело, пропали морщины, глаза стали просто огромными.
Мне ужасно хочется крови (зачёркнуто) кро (закрашено полностью)
Я подозвала Бердышку и сделала ему небольшой порез, но это лишь распалило жажду.
Боюсь, что\ldotst>>

<<29 августа 1929 года.
Всё отлично.
Сайры-хаэр "--- замечательный человек. 
Нужна человеческая кровь.
Да, это звучит ужасно, но её нужна какая-то капля.
Я объяснила ситуацию товарищам, и Тоня дала мне каплю своей крови.
Их поразила произошедшая во мне перемена.
Они тоже хотят вечной молодости.
Сайры-хаэр рассказал, что вечную молодость мы сможем поддерживать, только если рядом будет один ,,простой человек``.
Иначе мы все умрём от жажды>>.

<<6 сентября 1929 года.
Я в ярости.
Тоня и Надя, не посоветовавшись со мной, получили ,,посвящение``.
Они довольны "--- молодость возвращается к ним.
У Тони даже пропал шрам на животе.
Но что делать?!
Как мы вернёмся обратно?!
Как мы объясним свою ежемесячную жажду крови?!
Надю это не волнует.
Она сказала, что её муж и революцию вытерпел, и войну, и каплю крови для неё не пожалеет.
Чёртовы дуры!
Они не понимают, чем это обернётся!
Из-за этой странной заразы мечты Владимира Ильича пойдут прахом.
Люди разделятся на два класса, как тяйшхаэры "--- не знающие болезней сверхлюди и все прочие, служащие лишь источником здоровой крови.
Мы станем угнетателями, не лучше капиталистов.
Но товарищам плевать, они думают лишь о себе.
Если бы я могла\ldotst если бы я только могла вернуться и отказаться от этого ужасного дара!>>

\section{З.\,П.}

"--*Мне сказали, что вы знакомы с Костомаровой З.\,П.

"--*Да, знакома, "--- раздался из кресла чистый, совсем не старческий голос.
"--- Зачем она вам нужна?

"--*Насколько хорошо вы её знали?
Мне хотелось бы поговорить с вами о её дневнике.

Голос хрустально усмехнулся.

"--*Глах, так вот кто мой дневник-то стащил.
Ну-ка разверни меня, погляжу на этого молодца.

Одна из старух подошла и легко, словно пушинку, развернула кресло к Варенику.
Тот, не отрываясь, смотрел на сидящую в нём женщину.

"--*Это вы "--- Костомарова З.\,П.?

"--*Зинаида Павловна, "--- улыбнулась она.
"--- Не можешь поверить, что мне за сотню?

"--*Кое-что выдаёт возраст, "--- осторожно сказал Вареник.
"--- Ноги.

Женщина улыбнулась белыми, как волчьи клыки, зубами.

"--*Если бы Глаха носила что-то поприличнее этой стариковской хламиды, ты бы увидел, что её ноги утрут нос любой молодой. 
А я сломала спину и тридцать лет как прикована к кровати.
Какое-то время могла писать, сейчас руки даже не шевелятся.
Увы, от таких случайностей молодость не защищает.

Зинаида Павловна кивнула на пустое кресло.

"--*Садись, чего стоишь-мнёшься.
Я хоть и из дворян, да дворянства давно уж нет.
Вижу "--- дело серьёзное, раз меня искать решил.

\section{Судьба племени}

"--*Тяйшхаэры были совершенно особым племенем.
Многие историки мечтают найти что-то особенное, непохожее на других, а мне повезло.
На самом деле они были потомками финно-пермских племён, ближе к саамам, непонятно как оказавшихся в Саянах.
Это подтверждалось многим "--- легендами, ритуалами, языком, который походил на удивительно красивую смесь монгольских и финно-угорских корней.

"--*Что произошло потом? "--- спросил Вареник.
"--- Куда делись тяйшхаэры?

Зинаида Павловна помолчала.

"--*Через год после моего отъезда в Урянхае начались волнения.
К тяйшхаэрам пришли болезни.
Тиф выкосил почти всех <<непосвящённых>>, оставшиеся заболели сифой.
<<Посвящённые>> решили уйти в степи, чтобы выжить, но столкнулись с карательным отрядом чекистов.
Тяйшхаэры не поняли, что за железные палки в руках у их противника, и приготовились защищаться.
Мне это потом рассказали тувинцы, которые всё видели.
Они говорили, что с гор спустились могучие богатыри с тугими луками, одетые в богатые одежды из шкур и перьев\ldotst

Зинаида Павловна заплакала.

"--*Ты понимаешь, хороший мой, у тяйшхаэров была совсем другая одежда, не как у аратов.
Они ехали верхом на быках, а не на лошадях.
Их язык напоминал тыва лишь отдалённо.
Но чекисты этого не знали.

\section{Слава}

Вот она, истинная слава белого человека.
Неважно, на каком языке он говорит "--- на английском, русском, испанском или французском.
Чукчи были гордым вольным народом, а стали героями анекдотов.
Тяйшхаэрам досталось забвение.

\section{Кровь}

"--*Можешь дать мне крови? "--- жалобно спросила Вероника.

"--*Ника, опять?
Да что с тобой?

"--*Мне очень плохо, милый.

В этот раз Вареник почти не колебался.

"--*Я наберу в шприц и дам тебе.
Согласна?

"--*Да, согласна!

Вареник вытащил из аптечки завалявшийся пятикубовый, сорвал шуршащую обёртку, затем зубами затянул жгут на плече.
Вероника жадными глазами следила за каждым его движением.

Едва кровь коснулась её языка, как странный огонь в глазах угас без следа.
Вероника обессиленно рухнула на диван.

"--*Со мной что-то происходит.

Вероника подняла руку и посмотрела на неё.

"--*Рука стала тонкой.
Пропали шрамы на запястьях.
Лифчик стал больше на один размер.

Вареник схватил руку жены, осмотрел её и застыл в ужасе.
Шрамов действительно не было.
Более того, на ладонях почти пропали линии.
Кожа была гладкой и эластичной, словно у топ-модели.

"--*Милый, что со мной?

В глазах Вероники замерла растерянность.

\section{Странные смерти}

Вареник зевнул и потёр глаза.

"--*Я долго спал?
Ты меня не разбудил.

"--*Долго.
Я пытался.

"--*Сорян.

"--*Не извиняйся, я компенсировал всё твоей едой на день.

"--*Да ради Бога, я ещё приготовлю.
Нашёл что-нибудь?

"--*Нашёл кое-какие статьи.
Пару месяцев назад прошла череда странных смертей.
Все по одной и той же схеме "--- человек бледнеет, худеет, начинает бредить, затем впадает в ярость, затем кома и смерть.
Первоначально у них не наблюдалось ни признаков инфекции, ни нарушений работы органов, поэтому умирали они, как ты понимаешь\ldotst

"--* \ldotst в дурке, "--- закончил Вареник.
"--- Что за статьи?

"--*Одна статья из новосибирской газеты, одна запись на барабинском паблике ВКонтакте, ещё один похожий случай в Новокузнецке сняли на видео.
Видео называется <<ВАМПИР НАПАДАЕТ НА ЛЮДЕЙ!!!>>

"--*Вампир?

"--*Ну, тот, кто снимал, почему-то решил, что этот псих хочет выпить крови.
А, вот.
Ещё психиатр-нарколог из Новосибирска на Башорге, практически повторяет историю из газеты.
Сразу несколько незнакомых друг с другом людей начали бросаться на прохожих с разницей в пару дней.
Он предположил, что в городе появился какой-то новый синтетический наркотик.

"--*Сможешь узнать, в какую больницу их повезли?

"--*В третью.
Я уже там был днём и даже выяснил, кто, скорее всего, написал на Башорг.
Но поговорить не выйдет.

"--*Почему?

"--*Он поссорился с женой, выпил, видимо, ну и схватил инсульт.

"--*Умер?!

"--*Выжил.
Но лучше бы умер.

\section{Мультиверсум}

Шенкерман тупо хлопал глазами.
Строчки кода сливались в цветастую картину: оператор цикла трудолюбиво крутил мельницу, скрипя итератором;
оператор выбора придирчиво осматривал переменные, как меняла монеты;
красиво курил, ожидая своей очереди, красавец \verb|return|.
А вот и один из багов, указанных в репорте.
Шенкерман почти видел, как эта жирная сволочь сидела на коде, смяв и раздавив хрупкие перекрытия вычислений.
Аналитическая геометрия.
Статистика.
Вероятность.

Большинство считает, что монетка падает лишь одной стороной вверх.
Математики знают вероятность выпадения одной из сторон.
Улыбчивый калека Стивен Хокинг вообще утверждает, что монетка падает и орлом, и решкой "--- в параллельных реальностях.

У Шенкермана было ощущение, что орёл и решка из параллельных реальностей встречаются друг с другом "--- или даже с самими собой.
Иногда он встречал самого себя, бросившего универ или оставшегося с той самой девушкой.
Иногда люди, с которыми он так и не познакомился, почему-то оборачивались ему вслед.
А у его вдовы при встрече с ним ёкало сердце.

Шенкерман любил теорию мультивселенной так же, как и теорию вероятности.

<<Интересно, какова вероятность того, что меня на этой неделе уволят?>> "--- промелькнула странная мысль.

<<Единица>>, "--- ехидно сказал чей-то голос.

<<Да пошёл ты>>, "--- обиделся Шенкерман и снова занялся кодом.

\section{Помощь}

Вокруг цвели деревья, жужжали насекомые, пели птицы, но Шенкерман видел молчаливый, холодный и белый покров зимы.

<<Как же люди одиноки в этом мире, "--- думал Шенкерман.
"--- Вся жизнь напоминает обучение нейросетей.
Место рождения, родители, тело, общество "--- случайные исходные данные, непонятные сигналы, с которыми что-то нужно сделать, чтобы доказать свою пригодность>>.

<<Тебе ли жаловаться на семью! "--- загремел в голове голос матери.
"--- Другие руки распускают, а то и вовсе нищие алкаши!
Мы тебе и образование дали, и одели-накормили\ldotse>>

Далее шёл длинный список заслуг матери перед Шенкерманом.

<<И всё-таки я от вас сбежал, "--- грустно подытожил Шенкерман.
"--- Плоть от плоти вашей, плод воспитания вашего.
Продолжайте винить дьявола и плохую компанию>>.

Шенкерман вдруг с какой-то нежностью вспомнил Вареника.
Почти сразу раздался звонок, вернув Шенкермана в жестокую реальность "--- к работе, пустой квартире и ноющей шее.

"--*Да.

"--*Володя, мне нужна твоя помощь.

"--*Вареник, у меня послезавтра дедлайн по проекту.
Оставить его мне не на кого.

"--*Но\ldotst

"--*Слушай, меня это всё уже заебало.
Благотворительность-хуительность, бабки эти ваши странные\ldotst
У меня работа, от которой зависит, буду я кушать или нет.
Всё прочее "--- не более чем хобби.
Отъебись хотя бы на неделю.

"--*А жизнь Вероники "--- тоже хобби?

Шенкерман поперхнулся.

"--*Что?

"--*Я нашёл старые записи об экспедиции на Саяны.
Скорее всего, она подхватила редкий вирус.
Вернее, её заразили.

"--*А я тебе кто "--- вирусолог?
Вези её в больницу!

"--*Нельзя.

Шенкерман судорожно вздохнул.
Вареник плакал в трубку.

"--*Блядь, Вареник!
Если меня уволят, ты труп, тебе ясно?
Буду через двадцать минут.

\section{Увольнение}

"--*Вам есть что сказать?

Шенкерман молчал.
Можно было придумать тысячу причин, почему проект не был готов.
Перебои со связью, обновления стороннего ПО.
Но бессонная ночь высосала из него все силы.

"--*Жена друга заболела, "--- честно ответил Шенкерман.

Алексей Анатольевич вздохнул.

"--*Клиент потребовал заплатить неустойку в размере четырёхсот тысяч рублей, угрожая оглаской.
Плюс небольшая сумма за время простоя серверов.
Как Вы понимаете, для SibDeepTec'а это ощутимые финансовые и репутационные потери.
Согласно ТК РФ, мы не имеем права взыскивать с Вас более месячного оклада.
Тем не менее, учитывая, что Вы полностью признаёте свою вину, мы рассчитываем на то, что Вы добровольно погасите хотя бы половину долга.
В этом случае мы дадим Вам возможность написать заявление по собственному желанию и избежать проблем с трудоустройством в Новосибирске.

Шенкерман улыбнулся.
<<Умирать, так с честью>>.

"--*Алексей Анатольевич, не ранее чем месяц назад я просил, чтобы мне в помощь дали двух стажёров.
Но Вы решили сэкономить.
Мои, цитирую, <<профессиональные навыки вполне достаточны, чтобы работа была выполнена в срок>>.

"--*Сожалею, что оценка Ваших профессиональных способностей была ошибочной.
В данный момент мы говорили\ldots

"--* \ldotst об ошибке, "--- перебил Шенкерман.
"--- Именно о Вашей ошибке как руководителя, не принявшего меры на случай непредвиденных обстоятельств.
Увольняйте по статье.
Мой остаток за месяц пойдёт на уплату неустойки.
Остальное "--- только через суд.

Шенкерман повернулся и вышел вон, хлопнув дверью.

\section{Дорога домой}

Домой в Бердск Шенкерман ехал, глядя на пылающий закат в окне полупустой маршрутки.
Закат время от времени вырывался из плена высотных зданий, мостов, придорожных кафе и густого переплетения леса;
каждый его поцелуй вызывал у Шенкермана лёгкую улыбку.
Напротив сидела девушка, и закат заставлял искриться оранжевым её прямые каштановые волосы.
Лица Шенкерман так и не увидел.
Или просто не запомнил.

"--*Володь, "--- голос Вареника был едва различим в трубке, "--- мы с Никой на первоначальный взнос копили, деньги есть.
Возьми пятьдесят штук, на месяца два-три тебе хватит.

Шенкерман отказался.
Не смог.

Звонил директор SibDeepTec'а.
Угрожал подать в суд.

"--*Идите на хуй, "--- сказал ему Шенкерман и сбросил звонок.
Он знал, что до суда дело не дойдёт, и был прав.

Чтобы успокоиться, он написал ВКонтакте Саше Красноруцкой, с которой общался в универе.
Кажется, они познакомились на чьём-то дне рождения\ldotst нет, на рок-концерте.
Точно на рок-концерте.
Она тогда перебрала со спиртным, и он тащил деваху с розовой чёлкой до общаги, с трудом отбив её у каких-то сомнительных бритоголовых парней.
Шенкерман хорошо помнил, как дрожали его ноги "--- и отнюдь не из-за веса безвольного женского тела.
После окончания юрфака Саша состригла чёлку, вышла замуж за горнолыжного инструктора и уехала с мужем на Кубань.

<<Они пытаются тебя развести, дарлинг, "--- подтвердила Саша.
"--- Не ссы.
Будут проблемы "--- вот мой номер, звони сразу, у меня систер в прокуратуре по Дзержинскому району.
Если надо "--- сама приеду>>.

<<Добро иногда возвращается>>.

Шенкерман вдруг понял, что жутко хочет спать.

\section{Искренность}

Шенкерман любил Сашу.
Да, наверное, всё-таки любил.
Шансов у него не было никаких "--- её отношения были расписаны на десять парней вперёд.
<<Беги к мечте!
Прыгай через голову!
Подходи к недоступным девушкам!>> "--- этим пестрела лента ВКонтакте, заставляя его сверстников разбивать свои сердца и резать спины о чьи-то ножи.
Шенкерман умел реально оценить свои возможности и не пытался что-то предпринять.
Тем не менее, не проводить с ней время было выше его сил.

Как-то Шенкерман спросил Сашу:

"--*Скажи, как у тебя получается быть душой компании?

"--*Сэйм квесчен, "--- ухмыльнулась Саша.
"--- Полагаю, я просто искренняя.

"--*Искренняя? "--- удивился Шенкерман.
"--- И всё?

"--*Ну это так-то дохуя для нормального человека.
Мне повезло "--- я родилась отбитой наглухо.

"--*И как может помочь искренность?

"--*Душа компании "--- это уникальный чел, обладающий, как у нас на факультете любят говорить, \emph{харрризмой}.
Каждый чел по-своему уникален.
Но большинство лгут, пытаясь быть похожими на других.

"--*Ты хочешь сказать, что душой компании может быть каждый?

"--*Оф кос.
Надо лишь найти подходящую компанию.
А, ну или решить для себя, а надо ли тебе в принципе быть душой компании.
Это тоже хороший вопрос, потому что для незакомплексованного юнита плюшки сомнительные.

"--*Может, дело в характере?
Для искренности нужен характер.

"--*Так далеко в рассуждениях я не заходила, "--- призналась Саша.
"--- Передай пиво и хватит меня грузить.

\emph{Харрризма} стоила Саше дорого.
В её компании не было ни одного юриста, за исключением сестры;
она предпочитала тусоваться с гуманитариями и айтишниками.

\section{Жадность}

Нет, не прав был Иешуа Га-Ноцри.
Страшнейший из пороков "--- отнюдь не трусость. 
Жадность Иуды до денег, жадность Синедриона до власти отправили Иешуа на крест, а отнюдь не старый, сломленный службой, одиночеством и болезнью прокуратор.
Жадность "--- вот самый страшный порок, и едва ли кто-то усомнится в этом, глядя на сегодняшнюю Россию "--- безмерно богатую и бесконечно нищую, словно профессиональный попрошайка.

\section{Начало войны}

"--*Сюда уже едут, "--- сказала Зинаида Павловна.
"--- Они не должны никого найти.
Никого живого и тёплого.

"--*И что ты предлагаешь, Зинка? "--- осведомилась одна из женщин.

"--*Я предлагаю вам добровольно вернуть долг за долгую здоровую жизнь.

Поднялся гвалт.

"--*Да иди ты в пизду, полоумная!

"--*Она права!

"--*Да где права-то?
Бежать надо!

"--*Если кто-то хочет бежать "--- собирайте вещи, "--- сказала Зинаида Павловна.
"--- Я вам больше не командир.

Толпа разделилась;
больше двух третей ушли.

"--*Отлично.
Теперь окончательно выяснили, кто из нас порядочный человек, а кто нет.

"--*Так что делать-то, Зин? "--- спросила Глафира.

"--*Да известно что.
Не хотят долги возвращать "--- будем взыскивать.
Глаш, ты у нас фронтовая, можешь девкам объяснить, что да как?

Глафира обратилась к остальным:

"--*Нина, Маша "--- доставайте своё из загашника, смазать не забудьте.
Остальные "--- найдите ломы да ножи понадёжнее.
Начнём с этой общаги, с нижнего этажа.
Вы двое "--- на пожарную лестницу у входа, вы четверо "--- на заднюю.
Пока все на позициях не будут "--- молчим и глазами не сверкаем.
Всем ясно? 

\section{Война}

Вареник обернулся и увидел Глафиру.
От ворчливой старушки не осталось ничего "--- седые волосы лежали на плечах, словно у древней норны, голубые глаза горели неистовым огнём фанатика-берсерка.
Её рубаха и лицо были в крови, словно она разрывала тела врагов голыми руками.
Войдя в комнату, Глафира чинно повернулась к образам и трижды осенила себя крестным знаменем.

"--*Готово, Зин, "--- хмуро сказала она, вытирая нож.
"--- Тебя\ldotsq

"--*Не нужно, "--- остановила подругу Зинаида Петровна.
"--- Прошу тебя, возьми ещё грех на душу "--- помоги товаркам и себе.

"--*Да какой грех, Господи, "--- сказала Глафира.
"--- Коли не стану душегубкой, так Иисус Христос, Сын Божий, с меня вчетверо спросит "--- за то, что позволила жить лукавому.
Идём, девочки.

"--*Не, Глаха, я сама, "--- проскрипела Нина.
"--- Рука у тебя верная, что ни говори, но за себя сама отвечу, не обессудь.

\section{Пытки}

"--*Володьке с сердцем поплохело, он сейчас в реанимации, "--- сказал Вареник.
"--- Его в ментовке пытали, хотели выведать, что он знает про вирус из Зеленообска.
Если это правда, он тебя не выдал.

\section{Вероника}

"--*Я уволилась из <<Вектора>>, "--- пробормотала Вероника.
"--- Сказала, что устала и устроюсь в институте, поближе к дому и поспокойнее.

"--*Они не знают?

"--*Нет.

\section{Правила}

"--*Что это? "--- удивился Вареник.

"--*Правила.
Тебе их надо выучить и следовать им в точности.

"--*Это правила для семей, в которых живут ВИЧ-инфицированные.

"--*Именно.
Наш вирус определённо передаётся при контакте с кровью, а значит "--- правила те же самые.

Вареник пролистал брошюру.

"--*О, так мы можем\ldotsq

"--*Да, самец-засранец, мы можем.
Но только с презервативом.
Я попробую выделить антитела к вирусу и проверю его концентрацию во всех жидкостях организма.
Целоваться до той поры на всякий случай будем только так "--- чмок-чмок.

"--*А\ldotst

"--*Если захочешь пойти налево "--- я не возражаю.
Для меня сейчас важнее, чтобы ты был рядом.
Только не притащи в дом ещё один вирус, у нас и так перенаселение.

"--*Да не пойду я налево, чего ты!
Отгулялся уже!

"--*Я тебя за язык не тянула, дело твоё.
Кстати, я попросила Таню сделать мне справку, что я ВИЧ-положительная.
И ты то же самое будешь говорить людям, если понадобится.

"--*Ты головой думала, прежде чем просила?
Тебя могут уволить, от тебя друзья отвернутся!
Ты же знаешь, как тяжело жить с таким клеймом!

"--*Настоящие друзья никуда не уйдут.

Вареник кивнул.

"--*Ты будешь изучать его?
Сама, втайне от остальных.

"--*Только для того, чтобы обезопасить окружающих, "--- покачала головой Вероника.
"--- Он мне достаточно крови попортил.
Пусть сдохнет вместе со мной в безвестности.

\section{6}

Если за вечную жизнь приходится платить тем, что сам ты сотворить не в состоянии "--- это вечное рабство, а не вечная жизнь.

\section{Имя}

"--*Вареник, а как тебя всё-таки звать?

"--*Володь, ты издеваешься? "--- опешил собеседник.

"--*Я серьёзно, "--- Шенкерман выглядел слегка растерянным.
"--- Сколько тебя помню, ты всегда был Вареником.
А кто ты по паспорту-то?
Виталий?

"--*Валерий.
Валерий Николаевич.

"--*Ты потратил пять секунд на то, чтобы вспомнить, "--- заметил Шенкерман. 
"--- Как прозвище-то въедается.

"--*Неправда.
Ты меня просто ошарашил вопросом.

"--*А давно тебя по имени-отчеству величали-то?

"--*А вот чёрт знает.
В политехе вроде.
У нас препод был смешной, всех звал <<мастерами>>.
Я был у него <<мастер Сафронов>> или <<Валерий Николаевич>>.

"--*Он тебе шанс давал, Вареник, "--- ухмыльнулся Шенкерман.

"--*Какой шанс?

"--*Недостаточно родиться Валерием Николаевичем.
Им надо стать.

"--*Ещё немного "--- и я стану антисемитом, "--- предупредил друга Вареник.

"--*Да хоть в скинхеды иди, "--- отмахнулся Шенкерман.
"--- Ты данные по вирусу принёс или мне гороскоп составлять для поиска?

\section{Шоколадка}

Сладости и фрукты, которые принесла бабушка, уже давно закончились.
За окном царила летняя ночь, наполненная шелестом листьев и песнями сверчков, но сна не было ни в одном глазу.

Шенкерман в приступе нелепой надежды потянулся к рюкзаку.
Иногда по рассеянности он забывал про оставленную в кармане шоколадку или пачку печенья;
когда пропажа находилась, это было всегда сродни маленькому чуду.

В этот раз чуда не произошло.

\section{Смысл без любви}

В среду приходил человек в штатском.
Был не особенно обходителен "--- в ответ на вполне законное возмущение лениво ткнул в лицо Петру Алексеевичу ксиву.
Травматолог, скрипнув зубами, кивнул и велел медсёстрам вывести соседей Шенкермана в коридор.
Говорил человек в штатском долго и пространно, обращался исключительно на <<ты>>.
Припомнил и живущую в Канаде бабушку, и сестру "--- гражданку Израиля, вскользь упомянул и про заказанные из Китая микроконтроллеры "--- <<статья сто тридцать восемь точка один у ка эр эф>>.
Смысл букв и цифр до Шенкермана дошёл: будешь жаловаться "--- будут проблемы.

"--*Может, всё-таки напишешь заявление? "--- спросил Вареник.

"--*Я устал, "--- сказал Шенкерман.
"--- Кто я, в конце концов?
Простой айтишник.
Ни связей, ни денег.

"--*Ты не простой айтишник, Володька.
Ты гений.
Ты фактически дело шестидесятилетней давности раскрыл, понимаешь?
Раскрыл и разрулил не по закону, а по совести!
Это тебе надо в органах работать, а не этому чекисту!

"--*Да кто меня пустит в органы? "--- горько усмехнулся Шенкерман.

Вареник понял.
Больше они эту тему не поднимали.

Шенкерман лежал и думал, что же не так с этой больницей?
Ответ пришёл постепенно "--- безвкусные стены были вымерены до сантиметра, по нормам СанПиНов;
над каждой розеткой висела столь же безвкусная табличка, сделанная исключительно из-за требований вышестоящих органов.
А кафель, который покрывал стену позади раковины, совершенно не сочетался по расцветке с основной стеной.
В каждой детали был смысл, но не было ни грамма любви.

<<Как и во всей моей грёбаной жизни, "--- грустно подытожил Шенкерман.
"--- К чёрту.
Разберусь с переломом "--- и на Родину\ldotst
Хватит с меня этой российской действительности>>.

\section{Конец}

"--*Извините, Зинаида Павловна, что так вышло.

"--*Не извиняйся, милый.
Надо было, конечно, сразу обо всём рассказать\ldotst
У нас ещё были идеалы "--- служить Родине, пролетариату\ldotst да вот знала я, что с нами сделают, коль проболтаемся.

"--*Вы думаете, вас бы убили?

"--*Хуже, милый.
У моей матери, которая в Поволжье жила, как-то тётка была.
Поехала в город на работу, а в родной деревне авария.
Сошёл с рельс поезд с какими-то химикатами, деревня вспыхнула, как спичка\ldotst
Неизвестно, остался ли кто в живых, но сверху велели согнать машины да запахали пепелище. 
Приехала тётка "--- а деревни нет.
Поехала узнавать в город "--- а её никогда и не было.
В лечебницу забрали и мытарили там пять лет\ldotst

Вареник почувствовал, что плачет.

"--*И войну братоубийственную запахали.
Муж мой сам пошёл.
Мог не идти, да пошёл.
Написал как-то, что у немцев много солдат с дырками в спине.
От <<Вальтера>>.
Не трусы были, честные вояки.
А потом сам вернулся с пулей в спине "--- уже от ТТ.
Комиссар мне потом так и сказал: <<Скажи спасибо, красавица, что мы его тебе Героем Союза привезли.
Могли и по-другому>>.
А потом его победой "--- его Победой! "--- запахали сталинские грехи. 
И Чернобыль бы с Припятью так же запахали, коль можно было бы.
Ты уж мне поверь. 
Сейчас всё ж другие времена "--- интернет есть.
Сейчас не запашут\ldotst

Зинаида Павловна закашлялась.
Вареник поднёс ей чашку полуостывшего чая.

"--*Идём, милый, я тебя причешу, "--- Зинаида Павловна улыбнулась мужчине вымученной улыбкой.
"--- Не бойся, просто причешу.
Всё одно сегодня умру\ldotst
Глашка говорит, что Богу душу отдают люди, да мы-то не такие, мы люди советские, атеисты.
Знаешь, какая школа-то у нас была?
Ууу\ldotst
Да иди, не бойся.

Вареник, поколебавшись, подошёл и сел рядом с кроватью.
Зинаина Павловна взяла гребень "--- не железный, а обычный, деревянный.
Вареник закрыл глаза.
Он ощущал себя котом.
Как же всё-таки здорово, когда тебя чешут.

"--*Ну вот, "--- старушка отложила гребень в сторону и потрепала мужчину по щеке.
"--- Теперь иди.
Веронике привет от меня передавай.
Так и скажи ей "--- она последняя.
Будет благоразумной "--- последней и останется.

"--*Она уже хотела\ldotst

"--*Рано хотела.
Ну свели девки счёты с жизнью, ну и что с того?
Они-то пожили.
Вероника тоже пускай поживёт.
Всё одно, как ни крути "--- долгая молодость, здоровая жизнь, счастливая старость.
Всё одно дар, хоть и проклятый.
У вас ещё молодость-то прекраснее нашей "--- столько соблазнов, столько удовольствий, мир как на ладони.
Грех такую терять.

"--*А дети?

"--*Детей усыновите.
Много их нынче, сироток.
Кто, если не вы?

Вареник закинул сумку на плечо и проверил, на месте ли ключи от машины.

"--*Так значит, вы скоро умрёте? "--- обернулся Вареник к Зинаиде Павловне. 
"--- А откуда вы это знаете?

Но она уже не дышала.

\section{Чемодан}

"--*Ну, в общем, это.
Я совсем без жилья.
Можно я у вас хотя бы переночую?

"--*Да оставайся, конечно.
Ты билет хотя бы сдал?

"--*На хуй билет.

"--*Ну как на хуй! "--- буркнул Вареник.
"--- Деньги, и немалые.

"--*Есть вещи важнее денег.

"--*Что ты наделал, Володь, "--- грустно сказала Вероника.
"--- Там у тебя родичи, другая жизнь.
А что здесь?

"--*Кризис не закончится, "--- поддержал её Вареник.
"--- Будет только хуже.
Ещё и конец света на Новый год обещают\ldotst

"--*Да не будет никакого конца света, Вареник, "--- грустно улыбнулся Шенкерман.
"--- Хуже "--- будет.
Но это не конец.

"--*Надеюсь, ты ещё передумаешь.

"--*Возможно.
Но только не сегодня.

Вареник и Вероника улыбнулись в ответ.

"--*В этом доме тебе всегда рады.
Пойдём, я суп сварила.
С тыквой.

"--*Люблю тыкву, "--- сказал Шенкерман и задвинул ручку чемодана.
Ручка громко щёлкнула.

\section{P.\,S.}

В 2029 году Вероника Сафронова-Зима была номинирована на Нобелевскую премию по биологии за открытие двух ранее неизвестных механизмов старения человека.
Тогда же она призналась, что она все эти годы являлась носителем редкого ретровируса и тайно изучала его свойства.
Валерий Сафронов развёлся с Вероникой в 2021 году, но сохранил с ней дружеские отношения до самой своей смерти в 2028 году.
Владимир Шенкерман вынужден был эмигрировать в США в 2016 году;
он сделал неплохую карьеру в сфере информационной безопасности и прожил в Сиэттле всю оставшуюся жизнь.
